\documentclass[11pt,a5paper]{book}
\immediate\write18{cd .. && py makedictionary.py && py makeinterlinears.py}

\usepackage{geometry}
\geometry{margin=0.7in,a5paper}

\usepackage{fontspec}
%\setmainfont[Ligatures=TeX]{DejaVu Sans}
\setmainfont{Charis SIL}
%\setmainfont[BoldFont={Gentium Basic Bold}]{Gentium Plus}
\setmonofont{DejaVu Sans Mono}


\usepackage{vowel}
\usepackage{graphicx}
\usepackage{xcolor}
\usepackage{multirow}
\usepackage{setspace}
\usepackage{enumerate}

\usepackage{hyperref}

\usepackage{gb4e}
\noautomath
\usepackage{leipzig}

\usepackage{makecell}
\usepackage{tabularx}

\usepackage{changepage} % adjustwidth

\usepackage{parskip}



\definecolor{CoverTeal}{RGB}{30, 212, 181}
\definecolor{Black}{RGB}{0,0,0}
\definecolor{LightGray}{gray}{0.9}
\definecolor{AccentText}{RGB}{102, 0, 59}
\definecolor{GlobalTextColor}{gray}{0.1}

\makeatletter
\newcommand{\globalcolor}[1]{%
  \color{#1}\global\let\default@color\current@color
}
\makeatother



\AtBeginDocument{\globalcolor{GlobalTextColor}}





\renewcommand{\eachwordone}{\addfontfeature{Color=AccentText}\large}
\renewcommand{\eachwordtwo}{\tt\scriptsize}
\pretocmd{\glt}{\it}{}{}




\newcommand{\qcn}[1]{\textcolor{AccentText}{\large\textbf{#1}}}


\newcommand{\langname}{\qcn{ǂA}}
\newcommand{\langnamelong}{\qcn{ǂA Ṇṵĩ}}
\newcommand{\transl}[2]{\qcn{#1} \emph{#2}}

\newcommand{\grammsc}[1]{\textsc{#1}}
\newcommand{\CLF}[1]{\grammsc{CLF}\textsubscript{#1}}
\newcommand{\ERG}{\grammsc{ERG}}
\newcommand{\ABL}{\grammsc{ABL}}
\newcommand{\ACC}{\grammsc{ACC}}
\newcommand{\INTR}{\grammsc{INTR}}
\newcommand{\INSTR}{\grammsc{INSTR}}

\newcommand{\cmnt}[1]{\textcolor{red}{#1}}


\usepackage{colortbl}

\newcommand{\voidcell}{\cellcolor[gray]{.9}}



\begin{document}

\renewcommand{\arraystretch}{1.5}


\pagestyle{empty}
%\pagecolor{Black}
%\color{LightGray}

%\setlength{\parindent}{0pt}



\begin{center}

	\Huge

	\vspace{0.4in}

	\resizebox{0.6\textwidth}{!}{\textsc{THE PRACTICAL}}

	\vspace{0.2in}

	\resizebox{0.5\textwidth}{!}{ǂA ṆṴĨ}

	\vspace{1in}

	{\Large \textsc{ON PHONOLOGY, GRAMMAR,\\AND THE PREVENTION OF MOUTH INJURIES.}}

	\vfill

	{\Large \textsc{BY CANCRIZANS CANON}}

\end{center}

%\nopagecolor

\pagebreak


\section{Overview}

\begin{center}
	\resizebox{1.5\width}{!}{\qcn{ǃUmǃoi!}}\\
\vspace{1em}
\emph{(`Hello!')}
\end{center}

\vspace{2em}

\langnamelong~([ǂɑː ɳ̰m̩̰ḭ̃˧], \emph{`first words'} or \emph{`first language'}), or also just \langname, is, as much as it pains me to admit it, not a real language, but it does try to be. It is a naturalistic constructed language (conlang) that makes extensive use of complex \textbf{click sounds}, which are strange loud consonants you make by abusing your tongue like a cheapo suction cup. It is \emph{a priori}, meaning it is not based on any real-world language of the present or past. As for the purpose of designing this language, let's keep that a surprise.

While an original creation, the sound of \langname{} takes inspiration mainly by the beautifully intricate phonologies of the \textbf{Khoisan languages}, a group of many language families indigenous to southern Africa, which feature large inventories of decorated clicks, and strange phonation distinctions in vowels. In addition to that, there are sprinkles of other sound-looks I like, picked from languages such as Basque, Sanskrit, and one of my favourite families: Aboriginal Australian languages. \langname{}'s phonation-tone register system is similar to that of Burmese.

For what concerns grammar, \langname{} is typologically a mostly isolating language, analogous to Mandarin Chinese. It has a strict SVO (or better, AVP, as will be clear later) word order, it is \mbox{(split-)ergative}, and strongly head-final. It possesses almost no true ``grammatical particles'' in that very often they turn out to also double as regular nouns, like the relational nouns of Mayan languages.

This booklet should hold all information there is on both the phonology and grammar of \langname{}. However, this document, as all my conlanging stuff, may try to explain things in a bit more pedantic detail than what you'd expect if you're a big linguistics buff. It is definitely aimed more at casual readers that don't remember off the top of their head what a \emph{wh-in-situ} or an \emph{accessibility hierarchy}\footnote{Something to do with power and wheelchair ramps?} is; if you find parts of it make your eyes roll please go ahead and skip what's obvious to you.

This conlang is meant primarily to exist as spoken. Phrases in \langname{} in this book are presented in its specialised orthography, which is designed to prioritise ease of pronunciation, is explained in the following sections, and are displayed \qcn{like this	}. Instead, phonemic/phonetic transcriptions using the IPA are in /slashes/ and [square brackets] respectively.


\tableofcontents

\chapter{Phonology}

\langname{} makes, curiously, no phonemic distinction of voicing. It does, hower, distinguish \textbf{nasality} as a binary feature between oral and nasal, and \textbf{glottalization} from modal, to ``glottalic'', to full glottal closure. Glot\-talic phones involve some form of glot\-tal in\-ter\-ven\-tion (as in ejec\-tives or crea\-ky-voiced so\-no\-rants), while fully glottal sounds involve some (coarticulated) glottal closure. 

This structure can help navigate the oversized phonological inventory, from vowels to click consonants to non-click consonants, by elucidating these overarching symmetries:

%\begin{center}
%\renewcommand{\arraystretch}{2}
%\begin{tabular}{|c|c|c|c|c|c|c|}
%\hline
%\multirow{2}{*}{\makecell{Phonation\textrightarrow\\Type \textdownarrow}} & \multicolumn{2}{c|}{Modal} & \multicolumn{2}{c|}{Glottalic} & \multicolumn{2}{c|}{Full Glottal} \\ \hline
%  & Oral & Nasal & Oral & Nasal & Oral & Nasal\\ \hline \hline   
%Vowels & \makecell{Plain vowels\\ /a/} & \makecell{Nasal vowels \\/ã/} & \makecell{~\\Creaky-voiced\\ (laryngealised)\\ vowels -- /a̰/\\~} & \makecell{Creaky-voiced\\nasal vowels\\/ã̰/} & \\\hline
%Clicks & \makecell{Oral clicks \\/ǃ/} & \makecell{Nasal clicks\\/ᶰǃ/} & \makecell{Glottalized clicks\\/ǃˀ/} & \makecell{~\\Pre-nasal\\glottalized\\clicks /ᶰǃˀ/\\~} &  \\  \hline
%Non-Clicks & \makecell{Oral\\pulmonics \\/p/} & \makecell{Nasal\\ pulmonics \\ /m/} & Ejectives /qʼ/ & \makecell{~\\ Creaky-voiced \\ nasals {[m̰]} \\(only allopho-\\nically) \\~  } &\makecell{Glottal \\ stop \\ /ʔ/} \\ \hline
%\end{tabular}
%\end{center}


\begin{center}
\renewcommand{\arraystretch}{3}
\setlength\tabcolsep{1.8pt}
\begin{tabular}{c|ccc}
\makecell{Sound Type\textrightarrow\\ Phonation\textdownarrow} & Vowels & Clicks & Non-Clicks \\ \hline \hline
Modal Oral &  \makecell{Plain vowels\\ /a/} & \makecell{Oral clicks \\/ǃ/} & \makecell{Oral\\pulmonics \\/p/} \\
Modal Nasal &  \makecell{Nasal vowels \\/ã/} &  \makecell{Nasal clicks\\/ᶰǃ/} & \makecell{Nasal\\ pulmonics \\ /m/} \\ \hline
Glottalic Oral & \makecell{~\\Creaky-voiced\\ (laryngealised)\\ vowels -- /a̰/\\~} & \makecell{Ejective-contour\\Clicks\\/ǃqʼ/} & Ejectives /qʼ/ \\
Glottalic Nasal &  \makecell{Creaky-voiced\\nasal vowels\\/ã̰/} & \voidcell & \makecell{Creaky-voiced nasals\\ (only allophonic)\\{[m̰]} \\~  } \\ \hline
Full Glottal Oral & \makecell{``Broken''\\vowels\\/a̰ʔa/} &  \makecell{Glottalized clicks\\/ǃˀ/} & \multirow{2}{*}{\makecell{Glottal \\ stop \\ /ʔ/} } \\
Full Glottal Nasal & \makecell{Nasal broken\\vowels\\/ã̰ʔã/} &  \makecell{~\\Pre-nasal\\glottalized\\clicks /ᶰǃˀ/\\~} & \\
\end{tabular}
\end{center}

\section{Vowels}

Five vowel qualities are phonemically distinguished:

\begin{center}
    \begin{vowel}
        \putcvowel{\qcn{i}}{1}
        \putcvowel{\qcn{e}}{2}
        \putcvowel{\qcn{a}}{4}
	\putcvowel{(ɑ)}{5}
        \putcvowel{\qcn{o} /ɔ/}{6}
	\putcvowel{(o)}{7}
        \putcvowel{\qcn{u}}{8}
    \end{vowel}
\end{center}

though, arguably, \qcn{u} alternates between [u] and [o] realisations in somewhat free variation. [ɑ] exist as a ``backened'' version of \qcn{a} (see for example Section~\ref{sec:backvowelconstraint} on contextual backening). In addition, the following diphthongs are allowed and behave essentially as additional single vowel qualities for the purpose of registers and phonotactics:

\begin{center}
\qcn{au}, \qcn{ao} /aɔ/, \qcn{ai}, \qcn{oi} /ɔi/, \qcn{ui}, \qcn{oa} /ɔa/\footnote{Diphthongs are preponderantly closing; the only opening diphthong /ɔa/ is barely so.}
\end{center}

It should be noted that \qcn{ui} specifically could be seen as a backened or ``pre-backened'' version of \qcn{i}, so that, say, the sequence \qcn{ǃui} ought to be interpreted as the realisation of the phonetically impossible sequence \qcn{ǃi}. A similar but weaker relationship should exist between \qcn{ɔi} and \qcn{e}. This is paralleled in the distribution of open-to-close diphthongs which preferably appear in stressed syllables and frequently following ``backening'' clicks and consonants which wouldn't allow a front vowel in the same position. This rule isn't universal, however.

\subsection{Registers}

We anticipate that \langname{} has a concept of \emph{stress} or \emph{accent} whereby one syllable in a polysyllabic word (and occasionally in a syntactically close word sequence, like a noun phrase) is marked as \textbf{stressed}. This stress is expressed mostly through vowel length and in minor part volume, but not pitch.

Unstressed vowels may only be monophthongs. Stressed vowels, instead, may be a mono- or a diphthong, and in addition they carry one of six different phonations, or more precisely \textbf{six registers}, that is a phonation + tone combination:

\begin{center}
\begin{tabular}{c c c}
	Notation & Phonation & Tone\\
	\qcn{aa} & Oral Modal [a] & ˧\\
	\qcn{aã} & Nasal Modal [ã] & ˥˦\\
	\qcn{a̰a} & Oral Creaky [a̰] & ˨ (on back vowels) ˦ (on front vowels)\\
	\qcn{a̰ã} & Nasal Creaky [ã̰]& ˧\\
	\qcn{a̰ʼa} & Oral Broken [a̰ʔa] & ˨ (on back vowels) ˦ (on front vowels)\\
	\qcn{a̰ʼã} & Nasal Broken [ã̰ʔã] & ˥˦ \\
\end{tabular}
\end{center}

Creaky voice is realized as laryngealisation (creaky voice proper) or even pharyngealisation (``strident'' or ``sphyncteric''), with no phonemic distinction. Front vowels \qcn{ḛ ḭ} are more likely to be sphyncteric than proper creaky, and back vowels \qcn{o̰ ṵ} are more likely creaky than sphincteric. \qcn{a̰} is either, consistently with the frontness of its realisation. As a variant on the orthography, creaky voice may be marked with a grave accent \qcn{a̰a} \textrightarrow{}\qcn{àa} if combining diacritics are not allowed or unsupported.

The ``Broken'' register is better classed as a type of phonation rather than a cluster of pre-existing sounds. It consists of a crescendo of glottalisation up to a glottal stop that ``breaks'' the vowel. An off-glide of the same vowel quality escapes right after. This produces a characteristic stuttered sound, which one could imagine as the phonation equivalent of a contour tone. The off-glide is truly an inseparable part of the pattern; it is impossible to have broken diphthongs like \qcn{*a̰ʼi} and nasality is consistent across the glottal stop.

All of the vowel \qcn{e}'s nasal forms merge with those of \qcn{i}, so that there is no \qcn{*ẽ}.

The nasal forms of \qcn{u} are special in that the nasalisation and lip closure are strong enough that they are better transcribed as a syllabic /m/:

\begin{center}
	\qcn{*uũ} \textrightarrow \qcn{um} \textrightarrow /m/ [m̩˥˦] \\
	\qcn{*ṵũ} \textrightarrow \qcn{ṵm} \textrightarrow /m̰/ [m̩̰˧]\\
	\qcn{*ṵʼũ} \textrightarrow \qcn{mʼm} \textrightarrow /m̰ʔm/ [m̩̰ʔm˧]\\
\end{center}

Degrees of rounding of such syllabic /m/ are usually inconsequential.

 In the case of diphthongs, a single register is applied uniformly and a mid-swipe register change is not allowed (phonemically at least). In the orthography, the creaky voice diacritic is written on the first component and the nasal diacritic on the second (with the caveat that \qcn{*ũ} is replaced by \qcn{m}). The resulting table of vocalic phonemes is as follows:
%
%\begin{center}
%\renewcommand{\arraystretch}{1.5}
%\begin{tabular}{|c|c|c|c|c|c|c|c|c|c|c|c}
%\hline
%Stress & Register & \multicolumn{7}{c|}{Quality}\\ \hline \hline
%Unstressed & (unspecified) & \qcn{a} & \qcn{e} & \qcn{i} & \qcn{o} & \qcn{u} & \multicolumn{3}{c|}{}\\ \cline{1-9}
%\multirow{4}{*}{Stressed} & Plain  &\qcn{a(a)} & \qcn{e(e)} & \qcn{i(i)} & \qcn{o(o)} & \qcn{u(u)} & \qcn{au} & \qcn{ui} \\ \cline{2-8}
%&Nasal &\qcn{aã} &  \multicolumn{2}{c|}{\qcn{iĩ}} & \qcn{oõ} & \qcn{mm} & \qcn{am} & \qcn{uĩ} &  \multirowcell{2}{\shortstack{\textit{etc for other} \\\textit{diphthongs}}}\\ \cline{2-8}
%&Creaky  & \qcn{àa} & \qcn{èe} & \qcn{ìi} & \qcn{òo} & \qcn{ùu} & \qcn{àu} & \qcn{ùi}& \\ \cline{2-8}
%&Nasal Creaky &\qcn{àã} &  \multicolumn{2}{c|}{\qcn{ìĩ}} & \qcn{òõ} & \qcn{ùm} & \qcn{àm} & \qcn{ùĩ} \\ \cline{1-9}
%\end{tabular}
%\end{center}


\begin{center}
\renewcommand{\arraystretch}{1.5}
\begin{tabular}{ccccc}
\hline
\multicolumn{5}{|c|}{Unstressed (always short)}\\ \hline 
\qcn{a} /a/ & \qcn{e} /e/&\qcn{i} /i/&\qcn{o} /ɔ/&\qcn{u} /u/ \\ \hline 
\end{tabular}

\setlength\tabcolsep{3pt}
\begin{tabular}{ccccccc}
\hline
\multicolumn{7}{|c|}{Stressed}\\ \hline
\multicolumn{2}{c}{Plain} & \multirow{2}{*}{Nasal} & \multirow{2}{*}{Creaky} & \multirow{2}{*}{\makecell{Nasal\\Creaky}}& \multirow{2}{*}{Broken}& \multirow{2}{*}{\makecell{Nasal\\Broken}}\\
Short & Long &&& \\ \hline
\qcn{a} /a/&\qcn{aa} /aː/ & \qcn{aã} /ãː/ & \qcn{a̰a} /a̰ː/ & \qcn{a̰ã} /ã̰ː/  & \qcn{a̰ʼa} /a̰ʔa/ & \qcn{a̰ʼã} /ã̰ʔã/ \\ \hline
\qcn{e} /e/&\qcn{ee} /eː/ & \multirow{2}{*}{\qcn{iĩ} /ĩː/} & \qcn{ḛe} /ḛː/ & \multirow{2}{*}{\qcn{ḭĩ} /ḭ̃ː/}  & \qcn{ḛʼe} /ḛʔe/ & \multirow{2}{*}{\qcn{ḭʼĩ} /ḭ̃ʔĩ/}  \\ \cline{1-2} \cline{4-4}
\qcn{i} /i/&\qcn{ii} /iː/ &  & \qcn{ḭi} /ḭː/ &   & \qcn{ḭʼi} /ḭː/ &\\ \hline
\qcn{o} /ɔ/&\qcn{oo} /ɔː/ & \qcn{oõ} /ɔ̃ː/ & \qcn{o̰o} /ɔ̰ː/ & \qcn{o̰õ} /ɔ̰̃ː/& \qcn{o̰ʼo} /ɔ̰ʔɔ/ & \qcn{o̰ʼõ} /ɔ̰̃ʔɔ̃/ \\ \hline
\qcn{u} /u/&\qcn{uu} /uː/ & \qcn{um} /mː/ & \qcn{ṵu} /ṵː/ & \qcn{ṵm} /m̰ː/  & \qcn{ṵʼu} /ṵʔu/ & \qcn{mʼm} /m̰ʔm/\\ \hline
&\qcn{au} /au/ & \qcn{am} /ãm/ & \qcn{a̰u} /a̰ṵ/ & \qcn{a̰m} /ã̰m̰/ \\ \cline{2-5}
&\qcn{ui} /ui/ & \qcn{uĩ} /mĩ/ & \qcn{ṵi} /ṵḭ/ & \qcn{ṵĩ} /m̰ḭ̃/ \\ \cline{2-5}
&\qcn{oi} /ɔi/ & \qcn{oĩ} /ɔ̃ĩ/ & \qcn{o̰i} /ɔ̰ḭ/ & \qcn{o̰ĩ} /ɔ̰̃ḭ̃/ \\ \cline{2-5}
&\multicolumn{4}{c}{\emph{(etc. for remaining diphthongs)} } \\ \cline{2-5}
\end{tabular}
\end{center}

A stressed, plain register monophtong may also be \emph{predictably} long or short. Specifically, it will be short if word-final and / or following a glottal(ised) consonant \qcn{ʼ}, and it will be long otherwise. In the orthography, it will be accordingly written with a single or double letter. Instead, all unstressed vowels are short, while all diphthongs and all stressed non-plain vowels are long.


\section{Consonants}


The consonant phonemes of \langname{} are divided mainly by airstream mechanism into \textbf{ejectives}, \textbf{pulmonics} and \textbf{clicks}. They are all presented in the table that follows; the rest of this section will be dedicated to explaining its contents.

Cells in grey are unattested in \langname{} or impossible. Cells spanning multiple rows or column denote degrees of allophonic variation. Phonemese in parentheses are highly marginal.

\begin{center}
\renewcommand{\arraystretch}{1.9}
\setlength\tabcolsep{1.5pt}
\begin{tabular}{cc|cccccccc}
\multicolumn{2}{c}{} & Labial & Dental & Apical & Palatal & Lateral & Velar & Uvul. & Glott.\\ \hline 
\multicolumn{2}{c|}{Ejective} &\voidcell & \voidcell & \qcn{ṭʼ}  /ʈʼ/ & \qcn{cʼ} &  	 \qcn{tłʼ} /t͡ɬʼ/ & \voidcell & \qcn{qʼ} &  \\ \hline \hline
\multirow{5}{*}{\rotatebox[origin=c]{90}{Pulmonics}} & Stop & \qcn{p} & \multirow{2}{*}{\makecell{\qcn{ṯ}\\[.5em]/t͡s̪/}}  & \qcn{ṭ} /ʈ/ & 	\qcn{j}/ɟ/ & \multirow{2}{*}{\makecell{\qcn{tł}\\[.5em]/t͡ɬ/}}	 	&	\multirow{2}{*}{\qcn{k}} & \multirow{2}{*}{\voidcell} & \multirow{4}{*}{\makecell{\qcn{ʼ}\\/ʔ/}}  \\ \cline{2-3} \cline{5-6} 
& Affricate & \voidcell &  & \qcn{tṣ} /ʈ͡ʂ/ & 	\qcn{tš} /t͡ʃ/ & 	&  & \voidcell  	 \\ \cline{2-9}
& Trill & \voidcell & \voidcell & \qcn{r} /r/ & \voidcell & \voidcell & \voidcell & \voidcell \\  \cline{2-9}
& Nasal & \qcn{m} &  \makecell{\qcn{ṉ}\\ /n̪/} & \makecell{\qcn{ṇ}\\/ɳ/} & \makecell{\qcn{ñ}\\/ɲ/} & \qcn{l} & \multicolumn{2}{c}{\makecell{\qcn{ŋ}\\/ŋ \textasciitilde ɴ/}}  \\ \hline \hline
%& Glot. Stop & & & & & & &  \\ \hline \hline
\multirow{11}{*}{\rotatebox[origin=c]{90}{Clicks}} & Plain & \qcn{ʘ}  & \qcn{ʇ} & \qcn{ǃ} & \qcn{ǂ} & \qcn{ǁ}  \\ \cline{2-7}
 & Glottalised & \voidcell & \qcn{ʇʼ} &\qcn{ǃ} & \qcn{ǂʼ} & \qcn{ǁʼ}  \\ \cline{2-7}
 & Fric.-release & \voidcell  & \qcn{ʇx} &\qcn{ǃx} &\qcn{ǂx} & \qcn{ǁx}  \\ \cline{2-7}
 & Ejec.-release & \voidcell & \qcn{ʇqʼ} &\qcn{ǃqʼ} &\qcn{ǂqʼ} & \qcn{ǁqʼ}  \\ \cline{2-7}
 & Nasal &  \qcn{ɴʘ}   & \qcn{ɴʇ} &\qcn{ɴǃ} &\qcn{ɴǂ} & \qcn{ɴǁ}  \\ \cline{2-7}
 & Nasal Glott. &  \voidcell   & \qcn{ɴʇʼ} &\qcn{ɴǃʼ} &\qcn{ɴǂʼ} & \qcn{ɴǁʼ}  \\ \cline{2-7}
 & Nasal Fric.-rel. & \voidcell    & \qcn{ɴʇx} &\qcn{ɴǃx} &\qcn{ɴǂx} & \qcn{ɴǁx}  \\ \cline{2-7}
 & Pre-fricative & \voidcell    & \qcn{sʇ} &\qcn{šǃ} &\qcn{ṣǂ} & \qcn{łǁ}  \\ \cline{2-7}
 & Pre-fr. Glott. &   \voidcell  & \qcn{sʇʼ} &\qcn{šǃʼ} &\qcn{ṣǂʼ} & \qcn{łǁʼ}  \\ \cline{2-7}
 & Pre-fr. Ej.-rel. &   \voidcell  & \qcn{sʇqʼ} &\qcn{šǃqʼ} &\qcn{ṣǂqʼ} & \qcn{łǁqʼ}  \\ \cline{2-7}
& (Implosive-rel.) & \voidcell & \voidcell & (\qcn{ǃʛ}) & \voidcell & (\qcn{ǁʛ})  \\ \cline{1-7}
\end{tabular}
\end{center}


\subsection{Ejectives}

Consistently with the language-wide pattern of distinction of degree of glottalisation, \langname{} distinguishes a few ejective phonemes, all of which but \qcn{qʼ} are actually uncommon. Frontal (labial or dental) ejectives do not exist. The next three ejectives \qcn{ṭʼ, cʼ, tłʼ} parallel the pulmonic obstruents. [*kʼ] is notably missing; it is generally understood that this sound has merged into the palatal \qcn{cʼ}, which varies across [cʼ\textasciitilde{}kʲʼ]. 

The uvular \qcn{qʼ} is \emph{always} ejective, with no pulmonic counterpart. It originates from the lenition of clicks with ejective contour (class IV) where the click itself vanished leaving behind the lone uvular ejective.

\subsection{Pulmonics}	

It is useful, not only for the purpose of phonotactics, to classify pulmonics in terms of nasality. Specifically, we divide into \textbf{oral pulmonics} (obstruents and the trill), the \textbf{nasal pulmonics} (actual nasal and \qcn{l}), and finally the glottal stop has to be set aside as neither oral nor nasal\footnote{This is true in a more literal sense: since the glottis is behind the velum, a glottal closure is really insensitive to the lowering of the velum.}. It's necessary to imagine that \emph{phonemically} /l/ be a nasal consonant, even though phonetically it often is not, and in particular the nasal counterpart to /t͡ɬ/. This allows, for example, to explain sequences such as \transl{laã}{tongue}, whereas anywhere else an oral pulmonic + nasal vowel sequence is forbidden (see Section \ref{sec:syllables}).

The dentals are usually ``strongly dentalised'', often going as far as interdental, similarly to the situation in Australian Aboriginal languages, though this is not usually marked in transcription. The dental obstruent \qcn{ṯ} is tipically an interdental affricate, usually sibilant. Occasionally it may be a simple fricative (sibilant or or not), especially before front vowels, but this distinction is not phonemic.

The series marked as ``apical'' oscillate between the apical alveolar (like \qcn{r}) and the true subapical retroflex (as \qcn{ṭ} typically is). Here the stop and affricate are distinguished.

Similarly the palatal series varies in palatalisation from fully palatal \qcn{j} to the laminal palato-alveolar \qcn{tš}.

\qcn{ŋ}, while rare, is a true phoneme, and may also appear word-initially, see \transl{ŋàã}{woman} vs \transl{nàã}{to laugh}, and it must be seen as the nasal counterpart to \qcn{k}. Before front vowels it's always velar; before back vowels it alternates between velar /ŋ/ and the uvular allophone [ɴ].



There is significant allophonic variance associated specifically with the lack of phonemic value to voicing of consonants. Nevertheless, there are significant irregularities to keep in mind.

\begin{itemize}
	\item Labial or labiodental fricatives and affricates are unattested.
	\item Because of the phonemic distinction of nasality, the voiced stops [b], [d̪\textasciitilde{}ð], [ɖ] are allophonic for the voiceless stops /p/ /t͡s̪/ /ʈ/, instead of the nasals /m/ /n̪/ /ɳ/ as would be more typical. The same holds in palatal articulation where [c] and [ɟ] are the same phoneme, but in this case the voiced form is more common realization, and so it is marked as /ɟ/, or \qcn{j}.
	\item In guttural (velar-glottal) position, curiosly [ɡ] can substitute not only for /ŋ/ but also for the glottal stop /ʔ/. As for /k/, it may often affricate to [k͡x] or even [x] directly, especially before front vowels, while a back vowel may uvularise it to [q].
	\item /r/ is always a trill, never tapped (a tap is more likely to be perceived as a nasal). It is geminated always in medial position (which we reproduce in the orthography with \qcn{rr}), occasionally even word-initially.
\end{itemize}


\subsection{Clicks}

\langname's unique phonetic identity lies in its inventory of click consonants. While we will ultimately analyse each possible click sound as a separate phoneme, resulting in a disproportionately large inventory but simpler phonotactical rules, it must be understood that clicks are complex consonants best decomposed into many semi-independent features. We recall that a click is produced by enclosing a pocket of air in a surface between the tongue and the palate. It is necessary to fully seal this pocket to produce the click sound, and the mouth-palate sealing occurs along a circle passing trough a \textbf{rear point of contact}, laterally, and through a coronal \textbf{front point of contact}. In \langname{} the rear contact is \emph{tendentially} always uvular, while the front contact may be in several positions, similarly to pulmonic consonants. By downward movement of the tongue, the trapped air pocket is rarefied, akin to a suction cup. Finally, one point in the sealing is opened and air violently rushes into the pocket. The corresponding implosion produces the loud sound of the click. We thus may begin to list some parameters that may change in the production of the click and which may affect the sound:

\begin{itemize}
	\item The location in the mouth where the sealing is opened; this is what is referred to as the \textbf{point of articulation} of the click.
	\item The opening of the velum and simultaneous airflow through the nose, i.e. \textbf{nasality} (or better, pre-nasalisation).
	\item The closure of the glottis simultaneously with the click, i.e. \textbf{glottalisation}.
	\item The mode of release of the rear closure, after the click sound has been produced. These are called \textbf{contours} or \textbf{effluxes} and can be seen as coarticulation of the click with a uvular pulmonic.
\end{itemize}

Four points of articulations are distinguished in \langname{} (plus the rare bilabial):

\begin{itemize}
	\item  /ǀ/, written \qcn{ʇ}, is laminal dental. The sound is noisy and highest in pitch.
	\item \qcn{ǁ} is lateral. The release is lateral (typically only on one side) and far back in the mouth. The coronal position of the tongue does not affect the sound, which is noisy but lower in pitch than the dental, and with a characteristic `liquid' quality.
	\item \qcn{!} is the alveolar or alveolo-palatal click, for us also conveniently called apical, and the essential feature is that the tongue is pulled down (and back), resulting in a very clean and loud `pop' sound of lowest pitch.
	 \item \qcn{ǂ} is the palatal or palato-alveolar click. The tongue is flat and adhering to a wide area on the palate and the alveolar ridge; the tongue tip does not make contact. The tongue is pulled backwards (and slightly downwards), resulting in a higher-pitched, still clear `tick' sound.
	\item The rarer bilabial click \qcn{ʘ}. It usually begins as labial and moves to labiodental, and has a loud, very noisy sucking-like sound. A very limited set of manners of articulation is attested for \qcn{ʘ}, and it appears only in very few words. It likely originates from strong labialisation of other clicks.
\end{itemize}

Given a point of articulation, the language then distinguishes a total of ten different \textbf{manners of articulations} for each:

\begin{enumerate}[I]
\item \textbf{Plain} The click is oral, glottis open, and the back-release is tenuis.
\item \textbf{Oral Glottalized} The glottis is closed, and kept open for a short while after the click sounds. This may appear as the onset of the following vowel being delayed. The click is oral.
\item \textbf{Fricative-contour} The click is oral, glottis open, and the back release is into a uvular fricative [χ]. The frication is usually quite strong and raspy, granting these clicks an ``affricate'' sound.
\item \textbf{Ejective-contour} The click is oral, glottis open, and the back release is into a uvular ejective [qʼ].
\item \textbf{Nasal} The click is nasal. Because of the velar/uvular closure, a velar/uvular nasal [ŋ~ɴ] appears to sound throughout the click. The glottis must be open, back-release is tenuis.
\item \textbf{Nasal Glottalized} The glottis is closed, and kept open for a short while after the click sounds. This may appear as the onset of the following vowel being delayed. The click is nasal.
\item \textbf{Nasal + Fricative-contour} The click is nasal, glottis open, and the back release is a strong uvular fricative, marked [ʁ] as nasality is almost always accompanied by voicing.
\item \textbf{``Pre-fricative''} A fricative is sounded before the click closure. While this is not a true co-articulation, since the fricatives may not occur in \langname{} without a following click we class this series of clusters as separate consonant phonemes. The clicks are oral, glottis open, back release tenuis. Only a specific fricative precedes a certain point of articulation for the click; the combinations are /s̪ǀ/, /ʂǂ/, /ʃ!/, /ɬǁ/. The point of articulation of the fricative matches roughly with that of the frontal closure, but specifically in /ʂǂ/, /ʃ!/ the configuration of the tongue tip is \emph{opposite} that of the click; respectively switching from apical to laminal and laminal to apical. In a sense, the fast articulation of these prefricative clicks is agevolated by the momentum from this motion, which is mantained into the lingual motion to articulate the click. The ``crossed'' /ʂǂ/, /ʃ!/ involve a continuous, snake- or whip-like motion of the tongue, while the non-existent ``direct'' alternates /*ʃǂ/, /*ʂ!/ would be considerably more awkward.
\item \textbf{Pre-fricative + Glottalized} These clicks have a fricative onset, oral, glottal closure with delayed release of glottal stop.
\item \textbf{Pre-fricative + Ejective-contour} Fricative onset, oral, back release into [qʼ]
\item (rare/marginal) \textbf{Implosive-contour} These very rare clicks involve a released into a voiced implosive. They usually occur as alternates to plain clicks employed for humour or onomatopoeia. Only /*ǃʛ/ and /*‖͡ʛ/ are attested.
\end{enumerate}

All in all, the following 40 click phonemes, + 2 marginal bilabials exist:


\begin{center}

\begin{tabular}{|c|c|c|c|c|c|}
\hline Manner & \multicolumn{5}{c|}{Point of articulation} \\ \hline\hline
I &  	/ǀ/ &	/ǂ/ &	/!/ &	/ǁ/  & /ʘ/\\ \hline
II &  	/ǀˀ/ &	/ǂˀ/ &	/!ˀ/ &	/ǁˀ/ & \\ \hline
III &  	/ǀ͡χ/ &	/ǂ͡χ/ &	/!͡χ/ &	/ǁ͡χ/ &\\ \hline
IV &  	/ǀ͡qʼ/ &	/ǂ͡qʼ/ &	/!͡qʼ/ &	/ǁ͡qʼ/& \\ \hline
V &  	/ᵑǀ/ &	/ᵑǂ/ &	/ᵑ!/ &	/ᵑǁ/  &  /ᵑʘ/\\ \hline
VI &  	/ᵑǀˀ/ &	/ᵑǂˀ/ &	/ᵑ!ˀ/ &	/ᵑǁˀ/ &\\ \hline
VII &  	/ᵑǀ͡ʁ/ &	/ᵑǂ͡ʁ/ &	/ᵑ!͡ʁ/ &	/ᵑǁ͡ʁ/ &\\ \hline
VIII &  	/s̪ǀ/ &	/ʂǂ/ &	/ʃ!/ &	/ɬǁ/ &\\ \hline
IX &  	/s̪ǀˀ/ &	/ʂǂˀ/ &	/ʃ!ˀ/ &	/ɬǁˀ/ &\\ \hline
X & /s̪ǀ͡qʼ/ &	/ʂǂ͡qʼ/ &	/ʃ!͡qʼ/ &	/ɬǁ͡qʼ/& \\ \hline
(XI) & & & (/ǃ͡ʛ/) &  (/‖͡ʛ/) & \\ \hline
\end{tabular}

\end{center}

If we are willing to segment the click even more, a somewhat clearer picture emerges. Among manners, we can distinguish an ``onset'' feature, which may be plain, nasal, or pre-fricative, and a ``release'' feature, which may be tenuis, glottal, fricative, or ejective. The $3\times 4$ table that results makes it clear that all combinations except two are realised:

\begin{center}
\begin{tabular}{|c|c||c|c|c|c|}
\hline & &  \multicolumn{4}{c|}{Release}  \\ \hline 
& & ∅ & ˀ& χ/ʁ &  qʼ \\ \hline\hline
\multirow{3}{*}{\rotatebox{90}{Onset}} & ∅ & I & II & III & IV  \\ \cline{2-6}
& ᵑ & V & VI &  VII&  \\  \cline{2-6}
& F & VIII & IX & & X  \\  \hline
\end{tabular}
\end{center}

As for the two unattested manners, their absence may be explained by difficulty of production. The missing nasal-ejective clicks in particular would present the difficulty of switching from voiced to voiceless mid-click, or producing a fully voiceless nasal click, something that is certainly quite alien to \langname{} speakers.

In the orthography, the clicks are transcribed using the following dictionary:

\begin{center}
\begin{tabular}{c|c}
Phonemic & Orthography \\ \hline \hline
ǀ & \qcn{ʇ}\\
ᵑ* & \qcn{ɴ*}\\
s̪ǀ & \qcn{sʇ} \\
ʂǂ & \qcn{ṣǂ} \\
ʃ! &	\qcn{š!} \\
ɬǁ & \qcn{łǁ}\\
*ˀ & \qcn{*ʼ}\\
*͡χ / *͡ʁ & \qcn{*x} \\
*͡qʼ & \qcn{*qʼ}\\
*͡ʛ & \qcn{*ʛ}
\end{tabular}
\end{center}




\section{Phonotactics}

\newcommand{\stress}{ˈ}

Here and in the following, these abbreviations are employed to describe phonotactical rules:

\begin{center}
\begin{tabular}{cc}
	(\ldots) & Optional segment (may appear zero or one time)\\
	. & Syllable boundary \\
	C & Any consonant phoneme -- click, ejective or pulmonic.\\
	Ʞ & Any click consonant. \\
	P & Any pulmonic consonant. \\
	E & Any ejective consonant.\\
	M & A sonorant consonant (nasal or trill).\\
	V & \makecell{Any vowel, mono- or diphthong, \\stressed or unstressed, in any register} \\
 \stress{} & The following syllable is stressed \\
	v & An unstressed vowel (monophthong) 
\end{tabular}
\end{center}

\subsection{Back vowel constraint}\label{sec:backvowelconstraint}

A fundamental mechanical constraint applies to vowel qualities directly following specific clicks (backening clicks) and the ejective \qcn{qʼ}. Specifically, uvular articulations cause retraction of the tongue root, which makes it impossible to pronounce a front vowel directly after. In \langname, all clicks \qcn{ʘ ʇ ǃ ǂ ǁ} have uvular rear closure, and thus really release uvularly and cause tongue root retraction.

This back vowel constraint applies to

\begin{itemize}
	\item \qcn{qʼ}
	\item All non-glottalised clicks. (i.e., all groups except II and VI).
\end{itemize}

Glottalised clicks bypass the constraint because the glottal closure can be released with sufficient delay for the tongue to prepare in position for a front vowel, as in \transl{sʇʼe}{four}.

A backening consonant may not be followed by a front vowel \qcn{e} or \qcn{i}. In addition, \qcn{a} becomes [ɑ].  For a diphthong, the constraint applies to the starting quality of the glide, therefore \qcn{ui} may follow a backening click, as can \qcn{au}, though it will sound as [ɑu].

\subsection{Syllable structure and articulatory constraints}\label{sec:syllables}

\langname{} features a strict alternation of consonant and vowel, and thus a (C)V syllable structure. Generally, phonotactical restrictions appear as constraints related to the nasality and glottalisation features. The direction of consonant-vowel nasality interference is different for clicks and pulmonics, with the nasality of clicks interacting with that of the previous vowel and that of pulmonics with the following one. The precise rules are

\begin{itemize}
	\item in a VꞰ sequence, either both are oral or both are nasal.\footnote{Note that since a vowel preceding a click is always unstressed, this nasality will never be reported in the orthography.}
	\item in a PV sequence, P cannot be oral if V is nasal.
	%\item in an EV sequence, V is oral, unless E is \qcn{qʼ}, in which case V may be oral or nasal.
\end{itemize}

E.g.: the sequences /aǃ-/ and /ãᵑǃ-/ are possible, but /*ãǃ-/, /*aᵑǃ-/ are not possible; while the sequences /pa/, /ma/, /mã/ are allowed but /*pã/ can not occur.

For what concerns glottalisation, 

\begin{itemize}
\item a CV̰ sequence with a creaky voiced vowel will erase glottalization distinctions in the consonant C. This means that sequences like /ǃˀa̰/ and /ǃa̰/ are not phonemically distinct -- by convention we will transcribe the click without glottalisation. On the same line, ejectives and pulmonic obstruents are not distinguished before V̰, and we transcribe with the pulmonic by convention, except in the case of \qcn{qʼ} since it has no pulmonic equivalent.
\item A glottal stop followed by a creaky vowel ʔV̰ is indistinguishable from the lone vowel V̰. We chose to transcribe both broad IPA and orthography \emph{with} the glottal stop to preserve the simpler CV structure.
\item A sonorant will become itself creaky before a creaky vowel: MV̰ > M̰V̰, e.g. \qcn{màa} = /ma̰/ > [m̰a̰]. This is not marked at all in the broad transcription.
\end{itemize}



\subsection{Word structure and stress}


Due to the extremely minimal morphology, the vast majority of words appear uninflected. This uninflected form follows a very rigid scheme:

\begin{center}
\LARGE (v\textsubscript{0}).\stress{}CV\textsubscript{1}.(Pv\textsubscript{2})
\end{center}

In other words, we necessarily have a \textbf{main syllable} CV\textsubscript{1} which always stressed, and is composed of either a click, an ejective or pulmonic, and a vowel which, being stressed, may have any of the four registers, and be a mono- or diphthong. Optionally, one may have an unstressed \textbf{opening vowel} monophthong v\textsubscript{0}, and/or an unstressed \textbf{secundary syllable} with a pulmonic and a monophthong.

The possible word structures are named as follows:

\begin{center}
\begin{tabular}{cc}
	Monosyllabic &  CV \\
	Sesquisyllabic &   v.\stress{}CV \\
	Disyllabic & \stress{}CV.Pv\\
	Trisyllabic & v.\stress{}CV.Pv
\end{tabular}
\end{center}

The opening and secondary vowels, being unstressed, may not carry registers, and no distinction of phonation is made on them. However, phonetically the nasality of an opening vowel necessarily matches that of a main syllable click which follows as per the rules of Section~\ref{sec:syllables}, and this nasality is accordingly transcribed even if not phonemic.

\subsection{Irregular words and reduplication}\label{sec:redup}

Some special words break the patterns described thus far. A select few are lexicalised idioms and onomatopoeias. Most, however, are produced by one of the very few morphological processes of the language, which is \textbf{main-syllable reduplication}. This self-explanatory operation is used on adjectives and adverbs to mark comparatives, and on verbs to mark the applicative voice. The main syllable of the word is repeated, usually violating the word structure, exceeding the maximum number of syllable, and producing words with multiple clicks:

\begin{center}
\transl{ʇaala}{easily} \textrightarrow \transl{ʇaʇaala}{more easily}
\end{center}

More accurately it can be described as a reduplication C \textrightarrow C\textsubscript{1}vC\textsubscript{2} of the main consonant only, with the insertion of the epenthetic vowel v. Here is the full list of rules for reduplication:

\begin{itemize}
	\item v is always short / unstressed; the quality is the same as that of the main vowel if a monophthong and its starting quality if a diphthong. In pronunciation, and especially when C is a click or ejective v may be very short if not fully elided (voiceless or glottalised)
	\item if C is a pre-fricative click, only C\textsubscript{1} is pre-fricated.
	\item if C is nasal and/or glottalized, the nasality and/or glottal closure persist throughout the C\textsubscript{1}vC\textsubscript{2} sequence. Example: \transl{ɴǁui}{to jump} \textrightarrow \transl{ɴǁuɴǁui}{to jump on}, the vowel v (\qcn{u}) is nasal, though this is unmarked. In \transl{sʇʼi}{slender} \textrightarrow \transl{sʇʼʇʼi}{more slender}, the glottal closure is kept throughout and correspondingly the vowel is simply not written. This also applies to ejective C.
	\item if C has a uvular contour, then only C\textsubscript{2} gets the contour. Example: \transl{ǂxoiṭa}{strange} \textrightarrow \transl{ǂoǂxoiṭa}{more strange}
\end{itemize}

\subsection{Sandhi Rules}

Adjacent words that are syntactically close (generally, they are part of the same noun phrase, they are a noun-classifier pair, a de\-pen\-dant-post\-po\-si\-tion pair, an auxiliary-main verb pair, or simply part of a very short clause) are usually pronounced with no gap between them and are affected by \textbf{syntactical sandhi rules}.  These are assimilatory processes involving the vowel V that ends the preceding word, and the first sound of the following word. Depending on the latter, one may have vowel-click (VꞰ), vowel-vowel (VV), and vowel-pulmonic (VP) sandhi. Sandhi processes are never written in the orthography.

\emph{VꞰ sandhi} consists simply in V assimilating to the nasality of Ʞ, similarly to what would happen mid-word. This nasality will only be triggered if V is unstressed.

In \emph{VV sandhi}, the second vowel is an opening vowel and therefore always unstressed. Quality assimilation occurs according to the following scheme:

\begin{itemize}
\item if the first vowel is a diphthong, there is no assimilation and an epenthetic \qcn{ʼ} is inserted.
\item if the sequence VV describes a valid diphthong, assimilate to that diphthong.
\item \qcn{a-e} and \qcn{o-e} assimilate to \qcn{ai} and \qcn{oi} respectively.
\item \qcn{o-u} assimilates to \qcn{oo}
\item In all remaining cases (e.g. \qcn{u-a}) there is no assimilation and \qcn{ʼ} is inserted.
\end{itemize}

If there is assimilation, then the first vowel determines the register.

\subsection{Click-Pulmonic Harmony}

\textbf{Click-Pulmonic Harmony} is a specific kind of consonantal harmony that occurs in words that contain both a click and a secondary pulmonic, (v)ꞰVPv. At present, Ʞ-P harmony is mostly a phonological constraint on roots rather than a dynamic phenomenon given that there is no real morphology to speak of. Therefore, a speaker or a potential learner of the language need not worry about this restriction.

Ʞ-P harmony only concerns the four lingual clicks \qcn{ʇ ǃ ǂ ǁ}, and restricts the possible points of articulation of P, provided that P is coronal (i.e. labial and dorsals are unaffected). This is due to the difficulty of transitioning between certain too different articulations, especially with the added momentum from the click release. The pulmonic consonants that may follow each of the clicks are as follows:

\begin{center}
\begin{tabular}{|c|c|c|c|c|}
\hline
Preceding Ʞ & \qcn{ʇ} & \qcn{ǃ} & \qcn{ǂ} & \qcn{ǁ} \\ \hline \hline
Bidental & Yes & No & No & No \\\hline
Apical/Retroflex & No & Yes & No & Yes \\\hline
Laminal/Palatal & No & Yes & Yes & Rarely \\\hline
Lateral & Yes & Yes & Rarely & Yes \\ \hline
\end{tabular}
\end{center}

The occasional word violates the constraint -- for example \qcn{ǂxoiṭa} is Palatal \textrightarrow{} Retroflex, which is strange, coherently with the fact that it means \emph{strange}. 

Usually disharmonic tension is less likely to be problematic if a long vowel, especially a diphthong, separates the two offending consonants. Harmony does not usually act across word boundaries (likely because it would create too many ambiguities), but short vowels may lengthen to reduce strain, in particular if the words are syntactically close. For example, in the very name of the language, \langnamelong{}, the two adjacent words are syntactically related as a determiner-classifier pair, and the nominally short vowel \qcn{a} separates the disharmonic \qcn{ǂ} and \qcn{ṇ}. Therefore the vowel is usually lengthened in pronunciation, to ease the difficult movement. Hence the final pronunciation [ǂɑː ɳ̰m̩̰ḭ̃˧].

While weaker, a form of more general consonantal harmony applies when the main consonant is not a click. In particular bidentals are never followed by non-bidentals.

\section{Notes on Orthography}

The orthography of \langname{} is designed by prioritizing these guidelines:

\begin{itemize}
	\item Transparency: pronunciation should be easy and immediate to evince and reproduce. In particular, clicks should be well distinguished from pulmonics.
	\item Phonemic: it should be unambiguous, i.e. broad transcrip\-tion should be uniquely determined.
	\item Clarity: written text should be easily readable, avoiding too similar glyphs, or superscript and subscript glyphs.
	\item Portability: no combining diacritical marks should be used; only existing precomposed letters may be employed. (This is due to combining glyphs rendering improperly in many contexts). 
\end{itemize}

All are satisfied with the exception of the creaky voice low tilde diacritic on \qcn{a̰} and \qcn{o̰}, which violates portability\footnote{Curiously, \qcn{ḛ}, \qcn{ṵ} and \qcn{ḭ} are precomposed.} -- the grave accent alternative \qcn{àèìòù} can obviate in these cases.

The orthography uses conventional punctuation and most typesetting standards\footnote{There is no risk of confusing the alveolar stop glyph ǃ with the identical-looking, but distinct exclamation mark ! because phonotactics prevent clicks in syllable codas anyway.}. For what concerns capitalisation, for starting sentences or for proper names, I employ the typical Khoisan convention where the first \emph{capitalisable} character in the word is capitalised. Capitalisable characters include all latin letters including diacritics, the letter ŋ which becomes Ŋ\footnote{The shape of capital Eng may be widely different in different fonts. Shouldn't be a cause of concern.} the letter \qcn{ʇ} which capitalises as \qcn{Ʇ}, the click nasality letter \qcn{ɴ} which simply becomes \qcn{N} in uppercase; and the remaining letters (\qcn{ʼ ʘ ǃ ǂ ǁ}) don't capitalise. 

%Finally, the alphabetical order employed is as in the following table:
%
%\begin{center}
%\begin{tabular}{*{22}{|c}|}
%\hline
%Lowercase & \qcn{ʇ}&\multirow{2}{*}{\qcn{ʘ}}&\multirow{2}{*}{\qcn{ǃ}}&\multirow{2}{*}{\qcn{ǂ}}&\multirow{2}{*}{\qcn{ǁ}}&\multirow{2}{*}{\qcn{ʼ}}&\qcn{a}&\qcn{ã}&\qcn{à}&\qcn{b}& \qcn{c} &
%\qcn{č}&\qcn{d}&\qcn{e}&\qcn{è}&\qcn{i}&\qcn{ĩ}&\qcn{ì}&\qcn{j}&\qcn{k} &\qcn{l}\\
%Uppercase & \qcn{Ʇ}& & &  & & &\qcn{A}&\qcn{Ã}&\qcn{À}&\qcn{B}&\qcn{C} &
%                    \qcn{Č}&\qcn{D}&\qcn{E}&\qcn{È}&\qcn{I}&\qcn{Ĩ}&\qcn{Ì}&\qcn{J}&\qcn{K} &\qcn{L}\\ \hline \hline
%Lowercase & \qcn{ł} &\qcn{m}&\qcn{n}&\qcn{ṇ}&\qcn{ñ}&\qcn{ŋ}&\qcn{o}&\qcn{õ}&\qcn{ò}&\qcn{p}&\qcn{q}&\qcn{r}&\qcn{s}&
%                    \qcn{š}&\qcn{ṣ}&\qcn{t}&\qcn{ṭ}&\qcn{u}&\qcn{ũ}&\qcn{ù}&\qcn{x}\\ 
%Uppercase &\qcn{Ł} & \qcn{M}&\qcn{N}&\qcn{Ṇ}&\qcn{Ñ}&\qcn{Ŋ}&\qcn{O}&\qcn{Õ}&\qcn{Ò}&\qcn{P}&\qcn{Q}&\qcn{R}&\qcn{S}&
%                    \qcn{Š}&\qcn{Ṣ}&\qcn{Ṭ}&\qcn{ṭ}&\qcn{U}&\qcn{Ũ}&\qcn{Ù}&\qcn{X}\\ \hline
%\end{tabular}
%\end{center}

\chapter{Grammar}

\section{The Noun Phrase}

A noun phrase in \langname{} may consist of a single noun:

\begin{exe}
\ex
\gll uɴʇaã\\
wolf\\
\glt wolves / a wolf
\end{exe}

in which case the intended meaning is indeterminate, and of unspecified number (i.e. wolves in general, as one would intend in a phrase like \emph{`wolves are ferocious'}). If instead one would like to talk about one specific wolf, thus introducing determinacy, they would have to say

\begin{exe}
\ex
\gll uɴʇaã ṭa̰a\\
wolf \CLF{predatory animal}\\
\glt the wolf
\end{exe}

\qcn{ṭàa} is called a \textbf{noun classifier} (CLF), and it is specifically the classifier associated to predatory animals. There are hundreds of classifiers available for various categories of nouns; these categories do not have to be disjoint nor as general as standard noun classes in synthetic languages. When a CLF is used, the CLF is itself the head of the noun phrase, and the noun is a dependent that \emph{specifies} the general meaning of the classifier further (so the example may be translated more literally as \emph{`the predatory animal which is more specifically a wolf'}). This justifies why the CLF always \emph{follows} the classified noun, being that this language is strongly head-final.

Multiple CLFs may apply to the same noun under different circumstances, with subtler or more relevant differences in intended meaning depending on the situation. Rarely, a CLF choice may completely disambiguate a noun:

\begin{exe}
\ex
\gll ʇuli tła\\
mother/breast \CLF{woman}\\
\glt the mother
\end{exe}

\begin{exe}
\ex
\gll ʇuli ʇṵu\\
mother/breast \CLF{body part}\\
\glt the breast(s)
\end{exe}

Proper names are always determined and they \textbf{always} take a classifier. However, the choice of specific classifier is again up to the speaker, and may express some nuances of context, politeness, and relevant information:

\begin{exe}
\ex
\gll Nǃupaṇa ṉui\\
Nǃupaṇa 	\CLF{person}\\
\glt  Nǃupaṇa (a person of unspecified gender).
\end{exe}

\begin{exe}
\ex
\gll Nǃupaṇa 	tła\\
Nǃupaṇa 	\CLF{woman}\\
\glt Nǃupaṇa (the woman).
\end{exe}

A classifier is also triggered by numerals and partitives. When a numeral is used, the numeral is considered the head and the classifier its dependant, so the order is Noun-CLF-Numeral:

\begin{exe}
\ex
\gll ɴǃooʼo uṭu ɴǂoiči\\
chicken \CLF{bird} seven\\
\glt seven chickens.
\end{exe}

The explicit numeral \transl{ǃo̰o}{one} can be used to mark indeterminacy in situations where the presence of the classifier would be triggered anyway. For example:

\begin{exe}
\ex
\gll ʇuli tła ǃo̰o\\
mother/breast \CLF{woman} one\\
\glt a mother (but \textbf{not} a breast)
\end{exe}





\subsection{Possession and adjectives}

Dependants of a noun phrase precede the noun / pronoun they modify. This occurs always, independently of whether the dependant is used as a determiner or not\footnote{In general, there is no specific syntactic nor morphological marking for determiner phrases.}. Adjectives are always uninflected:

\begin{exe}
\ex
\gll ǁqʼa uɴʇaã ṭa̰a\\
big wolf \CLF{predatory animal}\\
\glt the big wolf
\end{exe}

Because of the effect of zero-copula, and the VS word order in intransitive clauses, there is no true distinction between an adjective and an intransitive/copular verb meaning to be that adjective (see Section~\ref{sec:copula}).

If a \emph{determined} noun phrase is placed as a dependant for another noun phrase, it marks an inalienable possessive. For example

\begin{exe}
\ex
\gll ŋa̰ã tła šǃʼa\\
woman \CLF{woman} teeth\\
\glt The woman's teeth
\end{exe}

Note that the classifier ensures \qcn{ŋa̰ã tła} is determined. If it was not, we would open ourselves to ambiguity:

\begin{exe}
\ex
\gll ŋa̰ã šǃʼa\\
woman teeth\\
\glt Teeth of a woman / *Female teeth
\end{exe}

Clearly, often context will be able to disambiguate. 

\subsection{Nominalisation}

\cmnt{todo}

\section{Alignment and Coordination}

\langname{} is always syntactically ergative. For intransitive clauses, with a verb V and a sole subject S, the verb always precedes the subject. For example

\begin{exe}
	\ex
%	\gll ĩᵑǁa̰ ᵑǃUpaɳa t͡ɬa \\
	\gll iɴǁa̰a Nǃupaṇa tła \\
	sleep Nǃupaṇa \CLF{woman}\\
	\glt `Nǃupaṇa is sleeping.'
\end{exe}

In a transitive clause, involving a verb V, an agent A and an object O, the order is \emph{fixed} as AVO:

\begin{exe}
	\ex
	\gll Nǃupaṇa 	tła 	iǃo̰orri 	šǃoiñe \\
		Nǃupaṇa 	\CLF{woman} 	eat 	meat\\
	\glt `Nǃupaṇa is eating meat'
\end{exe}

This rigid syntactical structure invites us to identify S and O as a single type of argument that always follows V, namely the Patient P, contrasting with agents A as a special role marked by preceding V. This syntactical alignment is therefore \textbf{ergative-absolutive} in nature. However, whereas a typical ergative language would provide a morphological way to mark Agents, i.e. an Ergative case, in \langname{} this does not usually occurr; the optional \ERG{} marker \transl{ʼa}{by, from} (which may equivalently also mark an Ablative) can be employed in special emphatic conditions (see Section~\ref{sec:topiccomment}):

\begin{exe}
	\ex
	\gll Nǃupaṇa 	tła  (ʼa)	iǃo̰orri  	šǃoiñe \\
		Nǃupaṇa 	\CLF{woman} (\ERG) 	eat 	meat\\
	\glt `Nǃupaṇa is eating meat'
\end{exe}

This overt marking is rare and considerably formal sounding; in the modern language it still doesn't allow for changing the word order except in a few idioms.

A transitive verb may be employed intransitively by omitting the Agent.

\begin{exe}
	\ex
	\gll iǃo̰orri 	šǃoiñe\\
eat 	meat\\
\glt `The meat is being eaten.'
\end{exe}

It is, however, ungrammatical to instead omit the Patient. Equivalently, a (lone) sentence may never finish on a verb. If we wanted to express a meaning alongside the lines of \emph{`ɴǃupaṇa eats (nothing specific)'}, we would need to perform a valency-changing operation that shifts argument so as to fill the Patient slot. An antipassive, marked by the auxiliary \qcn{uji}, does the job:

\begin{exe}
	\ex
	\gll  uji 	iǃo̰orri Nǃupaṇa 	tła  \\
	ANTIP eat	Nǃupaṇa 	\CLF{woman} 	\\
	\glt `Nǃupaṇa is eating (nothing specific)'
\end{exe}

We shall examine valency-changing operations in greater detail in Section~\ref{sec:valencychanging}.\\

We remark that it is possible to drop a repeated Patient in a coordinated clause, provided it is shared with a previous one. For example:

\begin{exe}
\ex
\gll !oono 	ji 	ɴǂuĩ 	 	ʼutła 	ṉoõ  ǂaãṇi ǂu\\
boy 	\CLF{child} 	kick 	ball 	\CLF{round tool} 	fly.away  and\\
\glt `The boy kicked the ball, and it flew away.
\end{exe}

In cases like these, the post-conjunction \transl{ǂu}{and} is preferred to the (here) equivalent pre-conjunction \transl{ʼai}{and, and then} because it prevents the clause from ending in a verb, though the second option would not be considered ungrammatical:

\begin{exe}
\ex
\gll !oono 	ji 	ɴǂuĩ 	 	ʼutła 	noõ  ʼai ǂaãṇi \\
boy 	\CLF{child} 	kick 	ball 	\CLF{round tool}   and	fly.away  \\
\glt `The boy kicked the ball, and it flew away.
\end{exe}

It is not, however, possible to omit a shared Agent in coordinated clauses, or to omit a Patient to be replaced with another clause's Agent and viceversa. For example, in \emph{`The boy kicked the ball and scored a point'} there is a shared Agent, and it is not possible to drop it in the coordinated clause in \langname{} like it is in English. A resumptive pronoun is necessary. And in \emph{`The boy kicked the ball and smiled'}, the boy is A in the first clause and P in the second, meaning that the boy's second appearence may not be dropped. (All of these example may of course be expressed with coordination and drop provided the right valency-changing is performed to make sure the coordinated arguments are always two Patients).

This behaviour, which persists under all conditions, concludes the other side of \langname's syntactical ergativity.

\subsection{Secundativity and ditransitives}

\langname{} lacks a type of complement that may be described as `Dative'. In a phrase involving a verb like \emph{give} (\textbf{ditransitive verb}), which involves some \emph{Donor} D giving a \emph{Theme} T to a \emph{Recipient} R, it is the Recipient which is treated as the direct object, while the Theme is placed in the instrumental (with postposition \qcn{ra}). For example

\begin{exe}
\ex
\gll Uǁa̰a ku ʇoã ałǁʼi ra  !ooṉo 	ji\\
Uǁa̰a \CLF{man} gift money INSTR boy \CLF{child}\\
\glt Uǁa̰a gifted money to the child. (lit. gifted the child with money.)
\end{exe}

Into this category of ditransitives fall not only verbs relating to giving, but also verbs concerning speaking, talking and telling -- the said thing is the Theme, and the addressee is the Recipient:

\begin{exe}
\ex
\gll Iǂxaaṇe ǃam ɴǁañe Aǃa̰uje ñḛʼe ɴǁʼam ra.\\
ancestor \CLF{spirit} tell Aǃa̰uje \CLF{elder man} story  \INSTR \\
\glt The ancestor told Aǃa̰uje a story.
\end{exe}

A construction is also possible where one employs a monotransitive verb as a ditransitive with the meaning of possession -- specifically the Theme is the original Patient, while the Recipient role is filled by the possessor of the original Patient. For example

\begin{exe}
	\ex
	\gll Nǃupaṇa 	tła ɴǁxape tłṵm ra  Aǃa̰uje ñḛʼe 	 \\
		Nǃupaṇa 	\CLF{woman} 	scold son \INSTR{} Aǃa̰uje \CLF{elder man}\\
	\glt Nǃupaṇa scolded Aǃa̰uje's son  (lit.: she scolded a son to him / he is getting his son scolded by her)
\end{exe}


\subsection{Causatives}

A \textbf{causer} of an action is seen as a ``super-Agent'', i.e. an argument placed even higher in a hierarchy of agency. If an agent is already present, the causer is placed in \textbf{ablative (ABL)}, which is to say paired with the postposition \qcn{ʼa}. Usually the ABL argument, both in the sense of causer and of literal ablative (motion from), is placed before the verb and possible agent, but its position in the sentence is somewhat more free than the closer arguments.

\begin{exe}
\ex
\gll Uǁa̰a ku ʼa Nǃupaṇa ǁa̰a ɴǂuĩ 	 	ʼutła 	ṉoõ\\
Uǁa̰a \CLF{man} ABL Nǃupaṇa \CLF{young woman} kick ball \CLF{round tool}\\
\glt Uǁa̰a made Nǃupaṇa kick the ball.
\end{exe}

However, in the absence of another true agent, the distinction between agent and causer is usually unimportant. Placing for example an argument in agentive position for an intransitive verb already communicates a meaning of causation:

\begin{exe}
\ex
\gll Uǁa̰a ku iɴǁa̰a Qʼoaʇṵu tła\\
Uǁa̰a \CLF{man} sleep Qʼoaʇṵu \CLF{young woman}\\
\glt Uǁa̰a made Qʼoaʇṵu sleep. (lit. he slept her)
\end{exe}

and, as seen before, forcing the argument into ablative position marks either a stronger emphasis for causation / volition, or focus for that argument (see again Section~\ref{sec:topiccomment}). For example:

\begin{exe}
\ex
\gll iɴǁa̰a Qʼoaʇṵu tła  Uǁa̰a ku ʼa\\
 sleep Qʼoaʇṵu \CLF{young woman} Uǁa̰a \CLF{man} ABL\\
\glt It was Uǁa̰a that made Qʼoaʇṵu sleep / Qʼoaʇṵu slept because of Uǁa̰a
\end{exe}

\section{Copula}\label{sec:copula}

\cmnt{Yeahh... I still have to write it.}

\section{Verbal voices}\label{sec:valencychanging}

Let us reprise, more in detail, the schema of a \langname{} verb phrase, with optional dependants in round brackets:

\begin{center}
(Causer) (Agent) Verb Patient (Oblique(s) + post.)
\end{center}

with no specific focus, now, on the word order. Used as such, a verb is said to be in \textbf{active voice}. When necessary, it is possible to redirect the arguments of the verb in different argument slots by employing a different verbal voice, marked by an auxiliary which goes before or after the verb. The simplest case, already seen, is the \textbf{antipassive}, only really sensible for a transitive verb, and formed by prepending \qcn{uji}; this is a \emph{demotion} in the agency hierachy, working in this manner:

\begin{center}
	Causer \textrightarrow Agent \textrightarrow Patient \textrightarrow Theme
\end{center}

In an antipassive sentence, the argument in agentive position (optional) has the meaning of causer, the one in patientive position (mandatory) has that of agent, and the instrumental oblique is the object. The purposes of this shift are several: it can be used to fill a patient gap, to express a transitive causative, or to relativize an agent (more on this in Section~\ref{sec:relative}).

\cmnt{Some examples needed.}

Antipassives may not be applied typically whenever an instrumental, especially in the role of theme, is present. A way to understand it is that instrumentals are part of the chain of agency described above, and the antipassive is attempting to demote the instrument to a position of lower agency that does not exist. I will describe shortly how to antipassivize a ditransitive, such as \emph{`Uǁʼàa gave money to his mother'} if we want to place \emph{Uǁʼàa} in the Patient position.

Another widely employed voice is the \textbf{applicative}. This is marked by the \textbf{main-syllable reduplication} (see Section~\ref{sec:redup}) of the verb and is used to \emph{promote} an oblique (of various types) to a patient. The chain is

\begin{center}
	Agent \textleftarrow Patient \textleftarrow Oblique
\end{center}

The applicative has thus some reminiscence of a passive, but it is restricted in that the original presence of an oblique argument to promote to patient is essential (it is ungrammatical otherwise). The applicative is a sacrifice of the information on the \emph{type} of oblique, since the postposition is lost, in exchange for transitivity of the verb, which may be necessary for relativisation.

Here's an example involving an oblique with the postposition \transl{iñi}{over}:

\begin{exe}
\ex
\gll ǁaũpe Uǁa̰a ku utłʼe iñi\\
walk Uǁa̰a \CLF{man} path on\\
\glt Uǁa̰a walks on the path
\end{exe}

With the applicative, one may produce the \textbf{transitive} verb \transl{ǁaǁaũpe}{to walk on} (but also potentially \transl{}{to walk in, into, with...}):

\begin{exe}
\ex
\gll Uǁa̰a ku ǁa\textasciitilde{} ǁaũpe utłʼe \\
Uǁa̰a \CLF{man} APPL\textasciitilde{} walk path \\
\glt Uǁa̰a walks on the path
\end{exe}

The ambiguity inherent in an applicative can be displayed by presenting an example of a different oblique, for example

\cmnt{examples}

\section{Pronouns}


\subsection{Personal Pronouns}

Exceptional within the language, the 1st and 2nd person pronouns are inflected, simultaneously for role (case), number, and clusivity.

\begin{center}
\begin{tabular}{|c|c|c|c|c|}
\hline
PNC & Refers to & \ERG & \INTR & \ACC \\ \hline \hline
1SG & Just the speaker &\qcn{ǃa} & \qcn{ja} & \qcn{eǃuũ}\\ \hline
1DU & The speaker + one addressee & \multicolumn{2}{c|}{\qcn{ǃxo̰o}} & \qcn{aɴǃxo̰o} \\ \hline
1PL.INCL & Speaker + addressee + others & \qcn{ʇṵupa} & \multicolumn{2}{c|}{\qcn{ṵuma}} \\ \hline
1PL.EXCL & Speaker + others (no addressee) &\qcn{ǃauṭa} & \qcn{jaṭa} & \qcn{eǃuũṭa}\\ \hline
2SG & Only one addressee &\multirow{2}{*}{\qcn{uǃʼui}} & \multirow{2}{*}{\qcn{utši}} & \qcn{ǃuũñi}\\ \cline{1-2} \cline{5-5}
2PL & Addressee(s) (+ others) &  &  & \qcn{ǃuũṭa}\\\hline
\end{tabular}
\end{center}

When used as the argument of an intransitive clause or a copular clause, a personal pronoun takes the \INTR{} case. If it's the agent of a transitive clause or the dependant of the proposition \qcn{ʼa} it takes the \ERG{}. In all other situations, meaning when used as the object of a transitive clause or as dependant to any other postposition, it takes \ACC{}. This implies that morphologically the language effectively has \textbf{tripartite alignment} in the first and second person (and, more precisely, nominative-accusative for 1DU and ergative-absolutive for the 1PL.INCL). Since the optional causative-ergative marker \qcn{ʼa} can be seen as a kind of (weak) morphological ergative marker that can instead be used on the 3rd person, one could argue that \langname{} is morphologically split-ergative, with the split occurring between the 2nd and 3rd person, while remaining always syntactically ergative\footnote{Minus the unnecessity of explicit case-marking, this is analogous to the alignment system of Dyirbal.}.

\subsubsection{Classifiers are the 3rd person pronouns}

Noun classifiers double as 3rd person pronouns. These are uninflected by case and number (though they may be optionally specified by an explicit numeral, identically to noun phrases). However, if one wants to reference a previously introduced noun that was determined by a classifier, one ought to use the same identical classifier as a pronoun. For example, if we refer through the noun phrase \transl{ɴǃupaṇa nui}{ɴǃupaṇa \CLF{woman}}, it would be considered ungrammatical then to later employ \transl{tła}{\CLF{woman}} with the same referent.

\cmnt{example}

\subsection{Demonstrative Pronouns}

\cmnt{todo}

\section{Relative Clauses}\label{sec:relative}

Being a primarily left-branching, ergative language, \langname{} is severely restricted in which positions are accessible for relativisation, a limitation that is obviated with the use of the aforementioned voices.

\textbf{Only the patient position may be relativised} -- meaning that the antecedent (the element that the relative clause describes) can only perform the role of patient in a relative clause. For example, amongst all these English examples

\begin{enumerate}
\item I saw the dog that was sleeping (Patient position)
\item I saw the dog that bit the cat (Agent position)
\item I saw the dog that my sister had gifted me (Instrumental position)
\item I saw the dog whose ears I find funny (Possessor position)
\end{enumerate}

only the first can be translated \emph{literally} into \langname{}, since `the dog' is Patient for the verb `sleep' in that case. The other examples have to be reworked with voice changing.

A simple (Patientive) relative clause is not marked with any special grammatical particle. It is simply placed before the antecedent with its own Patient omitted, constituting part of its noun phrase, and thus placing the antecedent itself in Patientive place for the relative clause. This entire noun phrase may then occupy any role in the \emph{main} clause. Here's an example where a main clause Patient is relativized, with the relative clause marked in [square brackets]:

\begin{exe}
\ex
\gll ǃa łǁa [ ǃa ǂqʼula ʼṵa ] ŋa̰ã (ṉui) \\
1S.\ERG{} see [ 1S.\ERG{} before meet ] woman (\CLF{person}) \\
\glt I saw the woman I had met.
\end{exe}

Generally such a determinative relative clause may trigger drop of the classifier, and in this case \qcn{nui} may be omitted, as we will do from now on. Determinacy is implied automatically.

We may also have the antecedent as Agent in the main clause:

\begin{exe}
\ex
\gll [ ǃa ǂqʼula ʼṵa ] ŋa̰ã łǁa eǃuũ \\
[ 1S.\ERG{} before meet ] woman see 1S.\ACC{} \\
\glt The woman I had met saw me.
\end{exe}

\cmnt{Proceed on voice changing and other positions.}

\section{Serial Verb Constructions}

\langname{} allows some kinds of \textbf{Serial Verb Constructions} (SVCs), whereby two or more verbs are chained together with no linking element in a single clause. Some of them are lexicalised (for example \qcn{uji ...} for the antipassive), in which case they can be understood better as auxiliary verbs, though fundamentally the spirit is that of SVCs, which are productive.

In the simplest kind of SVC two intransitive verbs are chained together sharing a patient -- this is called \textbf{intransitive patientive SVC}. The resulting combined intransitive verb has the meaning of performing the first action so that the second action may \emph{follow}, either just temporally or also causally. While not necessary, the first verb is usually one of motion. A practical example:

\begin{exe}
\ex
\gll ǂaã iɴǁa̰a ṇa̰ã ǂʼa \\
go sleep eland \CLF{large herbivores}\\
\glt The eland went to sleep (went so it could sleep / goes and sleeps)
\end{exe}

This construction may also help express TAM (Tense, Aspect, Mood) for intransitive verbs, using particular preceeding verbs, for example:

\begin{center}
\begin{tabular}{|l|c|}
\hline
Preceding Verb  & Translation of V. + ...\\ \hline\hline
 \transl{uʇum}{stand up} & be about to ..., will ..., be likely to...\\ \hline
\transl{ǁarra}{close (their) eyes} & refuse to ..., not intend to...\\ \hline
\transl{ǂaã}{go} & begin ..., go to do..., go there and ..., \\ \hline
\transl{tłʼoi}{exit, leave} & stop ..., finish ..., \\ \hline
\end{tabular}
\end{center}

We may not, however, serialize such intransitive verbs with a transitive verb, with the same types of meaning. For example, the following (with the presented intended meaning) is ungrammatical:

\begin{exe}
\ex
\gll *ṣǂa ku ǂaã ǃope ṇa̰ã ǂʼa \\
human \CLF{man} go kill eland \CLF{large herbivores}\\
\glt *The man went to kill the eland.
\end{exe}

because this construction would appear to attempt to share an argument between the patientive and agentive role in the SVC, which is not allowed by ergativity. You \emph{could} see the example as grammatical and translate it in the purely ergative sense as \transl{}{The eland went and got killed by the man}, but this \textbf{intransitive-first patientive SVC} construction is extremely uncommon, due to the unpleasant distance between the agent and the transitive verb it modifies. 

To communicate the meaning we originally wanted, which is \emph{`the man went to kill the eland'}, an astounding feature only possible thanks to \langname{}'s lack of ergative morphology is given by \textbf{chain SVCs}\footnote{The name serial verb construction is improper in this case since the verbs are not literally adjacent, but it still constitutes a monoclausal, polyverbal setup.}. In a chain SVC, an argument is placed, unmarked, \emph{inbetween} an intransitive and a transitive verb, in that order, followed by the object of the latter. The sandwiched argument acts as the patient to the first verb (given that it follows it), and as agent of the second (coming before it)\footnote{When a 1st or 2nd person pronoun is infixed as part of a chain SVC, as when translating \emph{`I went to kill the eland'}, the \ERG{} teform is commonly employed.}. This quite readily fixes the previous example:

\begin{exe}
\ex
\gll ǂaã ṣǂa ku  ǃope ṇa̰ã ǂʼa \\
go human \CLF{man} kill eland \CLF{large herbivores}\\
\glt The man went to kill the eland.
\end{exe}

Finally, we may also much more easily have regular SVCs where the first verb is transitive. If the second one is intransitive, then we have a \textbf{transitive-first patientive SVC}, where the transitive action causes the intransitive action:

\begin{exe}
\ex
\gll ǃXaoʼaã ku ɴǁxape ṯui ɴǃupaṇa  tła\\
ǃXaoʼaã \CLF{man} insult cry ɴǃupaṇa \CLF{woman}\\
\glt ǃXaoʼaã insulted ɴǃupaṇa and she cried (made her cry).
\end{exe}

If both are transitive, we have a \textbf{binary SVC}, where both agent and patient are shared. The meaning is more likely of temporal consecution than of causality, though this is not an absolute. Example:

\begin{exe}
\ex
\gll ǃXaoʼaã ku ǃope auǃqʼo ṇa̰ã ǂʼa\\
ǃXaoʼaã \CLF{man} kill skin eland \CLF{large herbivores}\\
\glt ǃXaoʼaã killed and then skinned the eland.
\end{exe}

In summary, the following kinds of SVCs are possible:

\begin{center}
\renewcommand{\arraystretch}{2}
\setlength\tabcolsep{1.3pt}
\begin{tabular}{ccccc}
\hline
\textbf{SVC type} & \textbf{V. 1} & \textbf{V. 2} & \textbf{Shared argument} & \textbf{Likely translation}\\\hline \hline
Intr. Patientive & intr. & intr. & Patient & Tense/Aspectual \\ \hline
Chain & intr. & tr. &\makecell{One infixed argument\\ acting as Patient\\and Agent respectively} & Tense/Aspectual \\\hline
Tr. Patientive & tr. & intr. & Patient & Causal \\\hline
Binary & tr. & tr. & \makecell{Both Patient and\\Agent separately} & Consecution \\  \hline
\end{tabular}
\end{center}

\section{Imperatives and Polarity}

\cmnt{To doo be doo}

\section{Interrogatives}

\cmnt{To doo be doo}

\section{Topic-Comment}\label{sec:topiccomment}

\cmnt{You guessed it}

\chapter{Corpus}

\section{The North Wind and the Sun}

\cmnt{This should be updated to the orthography changes.}

\newcommand{\prose}[1]{\begin{center}\begin{minipage}{0.8\textwidth}\large #1\end{minipage}\end{center}}

\prose{\qcn{Ʇaula Nǃòõ Uǃqʼa maã Jùu aje ǂu ełǁa iñi
nʇum nui ʼèe sʇaʇau ǁùuṇa.
Ṭurra, loõṇi ʇxoi onʇʼa oǃʼo qʼañi.
Maã aje ǂu ɴǃʼoirre nui ʼèe ǂa ʼa qʼañi ku ɴǁxòi upa ʇxoi ṭèe,
unʇu sʇaʇau tłìi šǃu nui.
Nǃòõ Uǃqʼa maã ǂxàuʼa tṣe ra tsùu ɴʘaã,
ṇam ča uji tsùu maã, qʼañi ku ʼa ṭèe onʇʼa ǃqʼati ɴǃxùu.
ʼai Uǃqʼa maã eña ṣǂqʼo.
ʼai ʼai, loõṇi ra čìiči Jùu aje, ʼai ǂu ku uǃqʼo ʇxoi ṭèe.
Nǃòõ Uǃqʼa maã łǁa sʇaʇau Jùu aʇe.}}

\prose{\emph{Once upon a time, the North Wind and the Sun were discussing over which one of them two was stronger. Suddenly, a vagabond wrapped in a warm cloak arrived to them.  The Wind and the Sun decided that the first of them that would make the vagabond take off the cloak, truly that one would have been the strongest.  The North Wind blew as strong as he could, but as the wind blew, the vagabond enveloped themselves in the cloak ever more. And so, the Wind gave up. But then, the Sun shone warmly, and so the vagabond took off their cloak. Thus, the North Wind saw that the Sun was stronger.}}

\begin{exe}
 \ex 
\gll {Ʇaula} {Nǃòõ} {Uǃqʼa} {maã} {Jùu} {aje} {ǂu} {ełǁa} {iñi}\\ 
once.upon.a.time North wind \CLF{weather} sun \CLF{celestial object} and conflict over\\ 
\glt Once upon a time, the North Wind and the Sun were discussing
\end{exe}

\begin{exe}
 \ex 
\gll {nʇum} {nui} {ʼèe} {sʇaʇau} {ǁùuṇa} .\\ 
two \CLF{people} PTV COMP\textasciitilde{}strong about {}\\ 
\glt over which one of them two was stronger.
\end{exe}

\begin{exe}
 \ex 
\gll {Ṭurra} , {loõṇi} {ʇxoi} {onʇʼa} {oǃʼo} {qʼañi} .\\ 
suddenly {} warm cloak envelop walk.into vagabond {}\\ 
\glt Suddenly, a vagabond wrapped in a warm cloak arrived to them.
\end{exe}

\begin{exe}
 \ex 
\gll {Maã} {aje} {ǂu} {ɴǃʼoirre} {nui} {ʼèe} {ǂa} {ʼa} {qʼañi} {ku} {ɴǁxòi} {upa} {ʇxoi} {ṭèe} ,\\ 
\CLF{weather} \CLF{celestial object} and decide \CLF{people} PTV first ABL vagabond \CLF{man} remove SUBJ cloak \CLF{clothing} {}\\ 
\glt The Wind and the Sun decided that the first of them that would make the vagabond take off the cloak,
\end{exe}

\begin{exe}
 \ex 
\gll {unʇu} {sʇaʇau} {tłìi} {šǃu} {nui} .\\ 
truly COMP\textasciitilde{}strong COND RES \CLF{people} {}\\ 
\glt truly that one would have been stronger.
\end{exe}

\begin{exe}
 \ex 
\gll {Nǃòõ} {Uǃqʼa} {maã} {ǂxàuʼa} {tṣe} {ra} {tsùu} {ɴʘaã} ,\\ 
North wind \CLF{weather} strain peak INSTR throw air {}\\ 
\glt The North Wind blew as strong as he could,
\end{exe}

\begin{exe}
 \ex 
\gll {ṇam} {ča} {uji} {tsùu} {maã} , {qʼañi} {ku} {ʼa} {ṭèe} {onʇʼa} {ǃqʼati} {ɴǃxùu} .\\ 
but while ANTIP throw \CLF{weather} {} vagabond \CLF{man} ABL \CLF{clothing} envelop self.ACC more {}\\ 
\glt but as the wind blew, the vagabond enveloped themselves in the cloak ever more.
\end{exe}

\begin{exe}
 \ex 
\gll {ʼai} {Uǃqʼa} {maã} {eña} {ṣǂqʼo} .\\ 
and wind \CLF{weather} surrender neck {}\\ 
\glt And so, the Wind gave up.
\end{exe}

\begin{exe}
 \ex 
\gll {ʼai ʼai} , {loõṇi} {ra} {čìiči} {Jùu} {aje} , {ʼai} {ǂu} {ku} {uǃqʼo} {ʇxoi} {ṭèe} .\\ 
and.then {} warm INSTR shine sun \CLF{celestial object} {} and and \CLF{man} peel.off cloak \CLF{clothing} {}\\ 
\glt But then, the Sun shone warmly, and so the vagabond took off their cloak.
\end{exe}

\begin{exe}
 \ex 
\gll {Nǃòõ} {Uǃqʼa} {maã} {łǁa} {sʇaʇau} {Jùu} {aje} .\\ 
North wind \CLF{weather} see COMP\textasciitilde{}strong sun \CLF{celestial object} {}\\ 
\glt Thus, the North Wind saw that the Sun was stronger.
\end{exe}


\chapter{Lexicon}

\section{Basic Classifier Taxonomy}

\cmnt{todopdeedoo}

\section{Greetings and idioms}

\cmnt{to do as well}

\section{Numerals}

\langname{} doesn't have a consistent way of expressing cardinal numbers larger than 24, and ordinals are even more severely under-developed, only rarely ever going as far as \emph{third}. The stable numerals are reported as follows, with * marking rare forms. The derivational patterns that can be evinced from many of these numerals are varied and chaotic. The constructions \transl{- ʼa ǃo̰o}{one from} and \transl{- ǂa}{next after} are used to create cardinals respectively one or more less than one with a simpler name, main-syllable reduplication may produce a number twice or thrice the original, and the almost unattested form of 22 seems to attempt a `second after' construction from 20, which itself is unstable to being represented either as \transl{pḛe}{digit} or \emph{`double ten'}, where 10 itself is \transl{aǃṵuma}{hand} (instead of it being assigned, more logically, to 5).

\begin{center}

\begin{tabular}{ccccc}
\hline
& Cardinal & Ordinal & & Cardinal\\ \hline \hline
1 & \qcn{ǃo̰o} & \qcn{ǂa} & 13 & \qcn{ŋum ǂa} \\ \hline
2 & \qcn{ɴʇum} & \qcn{ǃaaru} & 14  &  \qcn{ŋum ǃaaru} \\ \hline
3 & \qcn{eǂaaka} & \qcn{*ɴʇumrru} & 15 & \qcn{šǃoǃqʼoi} \\ \hline
4 & \qcn{sʇʼe} & & 16 & \qcn{ṉoṉoṯi} \\ \hline
5 & \qcn{šǃqʼoi} & & 17 & \qcn{ṉoṉoṯi ǂa} \\ \hline
6 & \qcn{aǁum} & & 18 & \qcn{aǁumñu}  \\ \hline
7 & \qcn{ɴǂoiči} & & 19 & \qcn{*pḛe ʼa ǃo̰o}\\ \hline
8 & \qcn{ṉoṯi} && 20 & \qcn{pḛe} (or \qcn{*aǃuǃṵuma})\\ \hline
9 & \qcn{ɴʇaaṯi} & & 21 & \qcn{pḛe ǂa} \\ \hline
10 & \qcn{aǃṵuma} & & 22 & \qcn{*pḛe ǃaaru} \\ \hline
11 & \qcn{ŋum ʼa ǃo̰o} & & 23 & \qcn{*ŋuŋum ʼa ǃo̰o} \\ \hline
12 & \qcn{ŋum}  & & 24 & \qcn{ŋuŋum} \\ \hline 
\end{tabular}

\end{center}



\section{Dictionary}

In the following dictionary, we report words in the standard orthography and in broad IPA transcription (in particular, no tones nor stress are marked, since they are fully predictable).

\begin{itemize}
	\item For nouns, we make suggestion of the most commonly used classifiers in [CLF \ldots].
	\item Some phrasal verbs are circumfixal, usually because they involve a lexicalized combination of a verb and a postposition. These are entered with dots \qcn{\ldots} to mark the space in which the Patient \emph{and} the oblique must be inserted.
	\item The subtler syntax of some verbs is clarified by expressing the action in terms of explicit arguments, namely (A) for agent, (P) for patient, (T) for theme (instrumental), and (ABL) for ablative/causer.
	\item For a language such as this one, alphabetical sorting is useless and cumbersome. I find it more practical and meaningful to present entries classified primarily by the main, i.e. first consonant instead.
\end{itemize}

%\setlength{\parskip}{2em}

\hyphenation{me-teo-ro-lo-gi-cal}
\hyphenation{phe-no-me-non}


\newcommand{\lettersection}[1]{
	\vspace{0.5em}
	\begin{center}
		\Large #1
	\end{center}
 	\par
}

\newcommand{\dictentry}[3]{\qcn{#1} - /#2/ \hangindent=0.4cm #3 \par }
\newcommand{\dictsense}[3]{$\bullet$~\emph{#1} #2 #3 }
\newcommand{\dictexample}[2]{\qcn{#1} \emph{#2}}
\newcommand{\dictclassifiers}[1]{\begin{scriptsize}[CLF #1] \end{scriptsize}}
\newcommand{\dictref}[1]{\qcn{#1}}
\newcommand{\dictsensesep}{\,\,}
\newcommand{\dictvariantof}[1]{\textleftarrow variant form of #1}

\newpage

\begin{multicols}{2}
\begin{singlespace}

\lettersection{ǁ}\dictentry{aǁum}{aǁmː}{\dictsense{card.\-num.}{six}{}}
\dictentry{ǁa}{ǁa}{\dictsense{card.\-num.}{few}{}}
\dictentry{ǁarra}{ǁarra}{\dictsense{v.\-intr.}{close one's eyes, not see, refuse to look}{}\dictsensesep\dictsense{v.\-intr.}{(in SVC) refuse to ..., not intend to ..., resist ...}{}}
\dictentry{ǁaũpe}{ǁãmpe}{\dictsense{n.}{foot}{\dictclassifiers{\dictref{ʇṵu}}}\dictsensesep\dictsense{v.\-intr.}{walk}{}}
\dictentry{ǁa̰a}{ǁa̰ː}{\dictsense{v.\-tr.}{hold}{}\dictsensesep\dictsense{clf.}{young women, recent mothers}{}}
\dictentry{ǁo̰i}{ǁɔ̰ḭ}{\dictsense{N/A}{NEG}{}}
\dictentry{ǁṵuṇa}{ǁṵːɳa}{\dictsense{post.}{about}{}}
\lettersection{ǁʼ}\dictentry{ǁʼu}{ǁˀu}{\dictsense{Ditransitive verb}{give, provide, (A) give (T) to (P)}{}}
\lettersection{ǁx}\dictentry{ǁxa}{ǁ͡χa}{\dictsense{n.}{penis}{\dictclassifiers{\dictref{ʇṵu}}}}
\lettersection{ǁqʼ}\dictentry{ǁqʼa}{ǁ͡qʼa}{\dictsense{adj.}{big, large}{}}
\dictentry{ǁqʼooña}{ǁ͡qʼɔːɲa}{\dictsense{n.}{crab}{}\dictsensesep\dictsense{n.}{lobster}{}}
\lettersection{ǁʛ}\dictentry{ǁʛa}{ǁ͡ʛ6a}{\dictsense{v.\-tr.}{scare, (P) be afraid of (A),}{}\dictsensesep\dictsense{n.}{fear}{\dictclassifiers{\dictref{ʇa}}}}
\lettersection{ɴǁ}\dictentry{iɴǁa̰a}{ĩᵑǁa̰ː}{\dictsense{v.\-intr.}{sleep}{}}
\dictentry{iɴǁoi}{ĩᵑǁɔi}{\dictsense{v.}{say}{}}
\dictentry{ɴǁañe}{ᵑǁaɲe}{\dictsense{Ditransitive verb}{(A) tell, recount, say (T) to (P)}{}}
\dictentry{ɴǁa̰ʼa}{ᵑǁa̰ʔa}{\dictsense{n.}{gold}{}}
\dictentry{ɴǁoi}{ᵑǁɔi}{\dictsense{v.}{can}{}}
\dictentry{ɴǁotṣo}{ᵑǁɔʈ͡ʂɔ}{\dictsense{n.}{door}{}}
\dictentry{ɴǁo̰õ}{ᵑǁɔ̰̃ː}{\dictsense{v.}{descend}{}}
\dictentry{ɴǁui}{ᵑǁui}{\dictsense{v.\-intr.}{jump}{}}
\lettersection{ɴǁʼ}\dictentry{ɴǁʼam}{ᵑǁˀãm}{\dictsense{n.}{story, tale}{\dictclassifiers{\dictref{ṇṵĩ}}}}
\lettersection{ɴǁx}\dictentry{ɴǁxape}{ᵑǁ͡ʁape}{\dictsense{v.\-tr.}{insult, berate, scold}{}}
\dictentry{ɴǁxo̰i}{ᵑǁ͡ʁɔ̰ḭ}{\dictsense{v.}{remove}{}}
\lettersection{łǁ}\dictentry{ełǁa}{eɬǁa}{\dictsense{n.}{conflict, discussion, disagreement, verbal fight}{}}
\dictentry{iłǁui}{iɬǁui}{\dictsense{n.}{milk}{\dictclassifiers{\dictref{ǂṵm}}}}
\dictentry{łǁa}{ɬǁa}{\dictsense{v.\-tr.}{see}{}}
\dictentry{łǁauʼi}{ɬǁauʔi}{\dictsense{adj.}{graceful, delicately beautiful}{}}
\dictentry{łǁoi}{ɬǁɔi}{\dictsense{clf.}{static formations, groups and sequences of inanimate objects, layouts, patterns and textures}{}}
\lettersection{łǁʼ}\dictentry{ałǁʼi}{aɬǁˀi}{\dictsense{n.}{money}{}}
\lettersection{łǁqʼ}\dictentry{łǁqʼa}{ɬǁ͡qʼa}{\dictsense{n.}{egg}{\dictclassifiers{\dictref{ṉoõ}}}}
\lettersection{ǂ}\dictentry{eǂaaka}{eǂaːka}{\dictsense{card.\-num.}{three}{}}
\dictentry{uǂuṇa}{uǂuɳa}{\dictsense{n.}{song}{}}
\dictentry{ǂa}{ǂa}{\dictsense{ord.\-num.}{first}{}}
\dictentry{ǂaã}{ǂãː}{\dictsense{v.\-intr.}{go, travel, move}{}\dictsensesep\dictsense{v.\-intr.}{(in SVC) begin ..., go to do..., go there and ...,}{}}
\dictentry{ǂaãñi}{ǂãːɲi}{\dictsense{v.}{flee}{}}
\dictentry{ǂaǂxa̰a}{ǂaǂ͡χa̰ː}{\dictsense{v.\-tr.}{snap, break (especially crack) in half}{}}
\dictentry{ǂa̰m}{ǂã̰m̰}{\dictsense{n.}{man}{}}
\dictentry{ǂoipe}{ǂɔipe}{\dictsense{adv.}{maybe.not}{}}
\dictentry{ǂootṣi}{ǂɔːʈ͡ʂi}{\dictsense{n.}{mountain}{}}
\dictentry{ǂoĩ}{ǂɔ̃ĩ}{\dictsense{v.\-intr.}{fly}{}}
\dictentry{ǂo̰õ}{ǂɔ̰̃ː}{\dictsense{clf.}{lids, lid-like objects, covers, doors}{}}
\dictentry{ǂu}{ǂu}{\dictsense{conj.}{and}{}}
\dictentry{ǂṵm}{ǂm̰ː}{\dictsense{clf.}{liquids, drops, rain, beverages}{}}
\dictentry{ǂṵʼu}{ǂṵʔu}{\dictsense{n.}{sun}{\dictclassifiers{\dictref{aje}}}}
\lettersection{ǂʼ}\dictentry{aǂʼui}{aǂˀui}{\dictsense{clf.}{wooden}{}}
\dictentry{ǂʼa}{ǂˀa}{\dictsense{clf.}{larger herbivores, elands, elephants, giraffes, etc}{}}
\lettersection{ǂx}\dictentry{ǂxa̰a}{ǂ͡χa̰ː}{\dictsense{v.\-tr.}{hit, strike with a loud sound}{}\dictsensesep\dictsense{v.\-tr.}{damage, hurt, offend}{}}
\dictentry{ǂxa̰uʼa}{ǂ͡χa̰ṵʔa}{\dictsense{n.}{effort, strain, force}{}}
\dictentry{ǂxoiṭa}{ǂ͡χɔiʈa}{\dictsense{adj.}{strange}{}}
\dictentry{ǂxoĩ}{ǂ͡χɔ̃ĩ}{\dictsense{post.}{through}{}\dictsensesep\dictsense{post.}{across}{}}
\lettersection{ǂqʼ}\dictentry{ǂqʼaĩ}{ǂ͡qʼãĩ}{\dictsense{v.}{know}{}}
\dictentry{ǂqʼula}{ǂ͡qʼula}{\dictsense{adv.}{before}{}\dictsensesep\dictsense{n.}{(anatomy) back, spine, buttocks}{}}
\lettersection{ɴǂ}\dictentry{aɴǂa̰i}{ãᵑǂa̰ḭ}{\dictsense{adj.}{every}{}}
\dictentry{uɴǂa̰aki}{mᵑǂa̰ːki}{\dictsense{v.\-intr.}{(S) climb}{}}
\dictentry{uɴǂoi}{mᵑǂɔi}{\dictsense{n.}{language, way of speaking}{\dictclassifiers{\dictref{ṇṵĩ}}}}
\dictentry{uɴǂṵu}{mᵑǂṵː}{\dictsense{rel.\-pr.}{which.ERG}{}}
\dictentry{ɴǂaã}{ᵑǂãː}{\dictsense{v.}{complain}{}}
\dictentry{ɴǂoa}{ᵑǂɔa}{\dictsense{v.\-intr.}{discover}{}}
\dictentry{ɴǂoitši}{ᵑǂɔit͡ʃi}{\dictsense{card.\-num.}{seven}{}}
\dictentry{ɴǂo̰õ}{ᵑǂɔ̰̃ː}{\dictsense{n.}{bed}{}}
\dictentry{ɴǂuĩ}{ᵑǂmĩ}{\dictsense{v.}{kick}{}}
\dictentry{ɴǂṵu}{ᵑǂṵː}{\dictsense{v.\-intr.}{die}{}}
\lettersection{ɴǂʼ}\dictentry{ɴǂʼi}{ᵑǂˀi}{\dictsense{Particle}{NEG, not, negates a preceding adverb, adjective or noun (not verbs)}{}}
\lettersection{ɴǂx}\dictentry{ɴǂxa}{ᵑǂ͡ʁa}{\dictsense{adv.}{always, often, it is a regular occurrence that}{}}
\lettersection{ṣǂ}\dictentry{ṣǂa}{ʂǂa}{\dictsense{n.}{human being, person}{}}
\dictentry{ṣǂum}{ʂǂmː}{\dictsense{n.}{knife}{\dictclassifiers{\dictref{ǃxatłe}}}}
\lettersection{ṣǂʼ}\dictentry{ṣǂʼo}{ʂǂˀɔ}{\dictsense{v.\-intr.}{sweat}{}\dictsensesep\dictsense{n.}{sweat}{\dictclassifiers{\dictref{ǂṵm}}}}
\lettersection{ṣǂqʼ}\dictentry{ṣǂqʼo}{ʂǂ͡qʼɔ}{\dictsense{n.}{neck}{}}
\lettersection{ǃ}\dictentry{aǃṵuma}{aǃṵːma}{\dictsense{n.}{hand}{}\dictsensesep\dictsense{card.\-num.}{ten}{}}
\dictentry{eǃum}{eǃmː}{\dictsense{pers.\-pr.}{me (ACC), to me}{}}
\dictentry{eǃumṭa}{eǃmːʈa}{\dictsense{pers.\-pr.}{1.PL.EXCL.ACC}{}}
\dictentry{iǃo̰orri}{iǃɔ̰ːrri}{\dictsense{v.}{eat}{}}
\dictentry{oǃao}{ɔǃaɔ}{\dictsense{adj.}{old}{}}
\dictentry{uǃa̰ama}{uǃa̰ːma}{\dictsense{n.}{misstep, mistake}{}}
\dictentry{uǃoõ}{uǃɔ̃ː}{\dictsense{n.}{year}{}}
\dictentry{uǃo̰i}{uǃɔ̰ḭ}{\dictsense{adj.}{jittery, irritable, violent, uneasy, startled}{\dictexample{Uǃʼui ʼa uǃòi ṇàã ǂʼa!}{You startled the eland!}}}
\dictentry{ǃa}{ǃa}{\dictsense{pers.\-pr.}{I (ERG), me, to me}{}}
\dictentry{ǃaala}{ǃaːla}{\dictsense{post.}{under, below}{}\dictsensesep\dictsense{post.}{moving by means of, travelling by}{}\dictsensesep\dictsense{n.}{palm (of hand), sole (foot)}{}}
\dictentry{ǃam}{ǃãm}{\dictsense{clf.}{spirits, ghosts, ancestors, the dead}{}\dictsensesep\dictsense{clf.}{trees}{}}
\dictentry{ǃauṭa}{ǃauʈa}{\dictsense{pers.\-pr.}{Us, excluding you (ERG)}{}}
\dictentry{ǃoa}{ǃɔa}{\dictsense{adj.}{idiot}{}}
\dictentry{ǃoi}{ǃɔi}{\dictsense{post.}{for the benefit of, for the purpose of giving to}{}\dictsensesep\dictsense{post.}{for the purpose/with the intent of going to, travelling to, or moving towards}{}}
\dictentry{ǃooja}{ǃɔːɟa}{\dictsense{preverb}{IMP.NEG}{}}
\dictentry{ǃoorro}{ǃɔːrrɔ}{\dictsense{n.}{urine}{\dictclassifiers{\dictref{ǂṵm}}}\dictsensesep\dictsense{v.\-intr.}{urinate}{}}
\dictentry{ǃooṉo}{ǃɔːṉɔ}{\dictsense{n.}{boy}{\dictclassifiers{\dictref{ji},\dictref{ṉui}}}}
\dictentry{ǃo̰o}{ǃɔ̰ː}{\dictsense{card.\-num.}{one, non-plural}{}\dictsensesep\dictsense{adj.}{lone, alone, unaccompained, unpaired}{}}
\dictentry{ǃo̰otło}{ǃɔ̰ːt͡ɬɔ}{\dictsense{n.}{vulva}{}}
\dictentry{ǃumña}{ǃmːɲa}{\dictsense{n.}{rain}{}}
\dictentry{ǃumñi}{ǃmːɲi}{\dictsense{pers.\-pr.}{2.S.ACC}{}}
\dictentry{ǃumṭa}{ǃmːʈa}{\dictsense{pers.\-pr.}{2.PL.ACC}{}}
\dictentry{ǃuuli}{ǃuːli}{\dictsense{n.}{celebration, party}{}}
\dictentry{ǃuũǃoi}{ǃmːǃɔi}{\dictsense{N/A}{hello, hi}{}}
\lettersection{ǃʼ}\dictentry{eǃʼaṇi}{eǃˀaɳi}{\dictsense{v.}{sing}{}}
\dictentry{oǃʼo}{ɔǃˀɔ}{\dictsense{v.\-intr.}{arrive (among others), join (Dat), meet up with others (Dat)}{}}
\dictentry{uǃʼui}{uǃˀui}{\dictsense{pers.\-pr.}{2.ERG}{}}
\dictentry{ǃʼa}{ǃˀa}{\dictsense{post.}{besides, to the side of, near}{}\dictsensesep\dictsense{n.}{hips, iliac crests}{\dictclassifiers{\dictref{ʇṵu}}}}
\dictentry{ǃʼiṉa}{ǃˀiṉa}{\dictsense{n.}{boat}{}}
\dictentry{ǃʼoã}{ǃˀɔ̃ã}{\dictsense{n.}{fingernail}{}\dictsensesep\dictsense{n.}{(of a location or a stretch in time) end, endpoint, boundary, finish, completion, last portion}{}}
\dictentry{ǃʼuulu}{ǃˀuːlu}{\dictsense{v.\-tr.}{bite}{}\dictsensesep\dictsense{v.\-intr.}{feel pain, especially itching of the skin}{}}
\lettersection{ǃx}\dictentry{oǃxo̰oji}{ɔǃ͡χɔ̰ːɟi}{\dictsense{v.}{get stuck, become unable to move or act}{}}
\dictentry{oǃxu}{ɔǃ͡χu}{\dictsense{n.}{walking cane}{}}
\dictentry{uǃxa̰m}{uǃ͡χã̰m̰}{\dictsense{v.}{concern}{}}
\dictentry{ǃxaje}{ǃ͡χaɟe}{\dictsense{v.\-tr.}{open}{}}
\dictentry{ǃxape}{ǃ͡χape}{\dictsense{adj.}{happy}{}}
\dictentry{ǃxatłe}{ǃ͡χat͡ɬe}{\dictsense{clf.}{blades, things with a sharp edge, teeth}{}}
\dictentry{ǃxa̰a}{ǃ͡χa̰ː}{\dictsense{n.}{house, hut}{}\dictsensesep\dictsense{n.}{roof}{\dictclassifiers{\dictref{ǂo̰õ}}}}
\dictentry{ǃxo̰o}{ǃ͡χɔ̰ː}{\dictsense{pers.\-pr.}{you and I}{}}
\lettersection{ǃqʼ}\dictentry{uǃqʼa}{uǃ͡qʼa}{\dictsense{n.}{wind}{\dictclassifiers{\dictref{maã}}}}
\dictentry{uǃqʼo}{uǃ͡qʼɔ}{\dictsense{v.\-tr.}{peel, scrape or remove a covering, protective layer, film, piece of clothing}{}}
\dictentry{ǃqʼao}{ǃ͡qʼaɔ}{\dictsense{n.}{clock}{}}
\dictentry{ǃqʼaṯi}{ǃ͡qʼat͡s̪i}{\dictsense{refl.\-pr.}{self.ACC, placed in P position, marks that (C) is also acting as (P).}{}}
\dictentry{ǃqʼoa}{ǃ͡qʼɔa}{\dictsense{v.\-tr.}{steal}{}}
\lettersection{ǃʛ}\dictentry{ǃʛoa}{ǃ͡ʛ6ɔa}{\dictvariantof{\dictref{ǃoa}}}
\lettersection{ɴǃ}\dictentry{ɴǃai}{ᵑǃai}{\dictsense{adj.}{similar to, akin to}{}}
\dictentry{ɴǃali}{ᵑǃali}{\dictsense{n.}{father}{\dictclassifiers{\dictref{ñḛʼe}}}}
\dictentry{ɴǃa̰a}{ᵑǃa̰ː}{\dictsense{n.}{fire}{}\dictsensesep\dictsense{v.\-tr.}{cook (on fire), roast}{}}
\dictentry{ɴǃooṭo}{ᵑǃɔːʈɔ}{\dictsense{n.}{chicken}{\dictclassifiers{\dictref{uṭu}}}}
\dictentry{ɴǃo̰õ}{ᵑǃɔ̰̃ː}{\dictsense{n.}{North}{}}
\dictentry{ɴǃo̰ʼo}{ᵑǃɔ̰ʔɔ}{\dictsense{v.\-intr.}{breathe}{}}
\lettersection{ɴǃʼ}\dictentry{ɴǃʼoirre}{ᵑǃˀɔirre}{\dictsense{v.}{decide}{}}
\lettersection{ɴǃx}\dictentry{aɴǃxo̰o}{ãᵑǃ͡ʁɔ̰ː}{\dictsense{pers.\-pr.}{me and you}{}}
\dictentry{ɴǃxa̰a}{ᵑǃ͡ʁa̰ː}{\dictsense{post.}{inside}{}}
\dictentry{ɴǃxṵu}{ᵑǃ͡ʁṵː}{\dictsense{adv.}{more}{}}
\lettersection{šǃ}\dictentry{išǃuka}{iʃǃuka}{\dictsense{N/A}{the very same}{}}
\dictentry{ušǃuupa}{uʃǃuːpa}{\dictsense{v.}{re-organize}{}}
\dictentry{šǃo}{ʃǃɔ}{\dictsense{Copulative verb}{be temporarily, be contingentially}{}}
\dictentry{šǃo ... iñi}{ʃǃɔ ... iɲi}{\dictsense{v.}{(smth) be over, be on top of}{}\dictsensesep\dictsense{v.}{(actions \& events) be involved in, act in, perform, be busy with}{}\dictsensesep\dictsense{v.}{lie on, lay down on, cover}{\dictexample{šǃo ku nǂòõ iňi}{he is lying on the bed}}}
\dictentry{šǃoiñe}{ʃǃɔiɲe}{\dictsense{n.}{meat}{\dictclassifiers{\dictref{la̰a}}}}
\dictentry{šǃu}{ʃǃu}{\dictsense{Resumptive marker}{RES}{}}
\lettersection{šǃʼ}\dictentry{šǃʼa}{ʃǃˀa}{\dictsense{n.}{tooth}{\dictclassifiers{\dictref{ʇṵu},\dictref{ǃxatłe}}}}
\lettersection{šǃqʼ}\dictentry{šǃqʼoi}{ʃǃ͡qʼɔi}{\dictsense{card.\-num.}{five}{}}
\lettersection{ʇ}\dictentry{iʇaã}{iǀãː}{\dictsense{v.\-intr.}{do nothing, be slacking, loiter}{\dictexample{Nǂxa iʇaã ǃXaoʼaã ku.}{ǃXaoʼaã is always slacking.}}\dictsensesep\dictsense{v.\-intr.}{lie down, be on the ground}{}}
\dictentry{uʇum}{uǀmː}{\dictsense{v.\-intr.}{stand up, stand up straight, get up}{}\dictsensesep\dictsense{v.\-intr.}{(in SVC) be about to ..., will ..., be likely to ..., plan to ..., intend to ...}{}}
\dictentry{ʇa}{ǀa}{\dictsense{clf.}{feelings, emotions, relationships, mental states}{}}
\dictentry{ʇaala}{ǀaːla}{\dictsense{adv.}{easily}{}}
\dictentry{ʇaula}{ǀaula}{\dictsense{adv.}{once.upon.a.time}{}}
\dictentry{ʇaõ}{ǀãɔ̃}{\dictsense{clf.}{orifices, bodily holes, openings, wounds}{}}
\dictentry{ʇoã}{ǀɔ̃ã}{\dictsense{Ditransitive verb}{(A) gift (T) to (P)}{}}
\dictentry{ʇuli}{ǀuli}{\dictsense{n.}{breast}{\dictclassifiers{\dictref{ʇṵu}}}\dictsensesep\dictsense{n.}{mother}{\dictclassifiers{\dictref{tła}}}}
\dictentry{ʇṵu}{ǀṵː}{\dictsense{clf.}{body parts}{}}
\dictentry{ʇṵupa}{ǀṵːpa}{\dictsense{pers.\-pr.}{1.PL.INCL.ERG}{}}
\lettersection{ʇʼ}\dictentry{iʇʼali}{iǀˀali}{\dictsense{n.}{barrier}{}}
\dictentry{iʇʼi}{iǀˀi}{\dictsense{clf.}{small animal}{}}
\dictentry{uʇʼule}{uǀˀule}{\dictsense{v.}{create, make, build, manufacture}{}}
\lettersection{ʇx}\dictentry{ʇxoi}{ǀ͡χɔi}{\dictsense{n.}{cloak}{}}
\lettersection{ʇqʼ}\dictentry{ʇqʼa}{ǀ͡qʼa}{\dictsense{n.}{night}{}}
\lettersection{ɴʇ}\dictentry{uɴʇaã}{mᵑǀãː}{\dictsense{n.}{wolf}{\dictclassifiers{\dictref{ṭa̰a}}}\dictsensesep\dictsense{adj.}{(of a person) unpredictably aggressive, pugnacious, cruel, dangerous}{}}
\dictentry{uɴʇu}{mᵑǀu}{\dictsense{adv.}{truly}{}}
\dictentry{ɴʇaaṯi}{ᵑǀaːt͡s̪i}{\dictsense{card.\-num.}{nine}{}}
\dictentry{ɴʇai}{ᵑǀai}{\dictsense{post.}{face}{}}
\dictentry{ɴʇa̰ʼã}{ᵑǀa̰ʔã}{\dictsense{n.}{womb, vagina}{\dictclassifiers{\dictref{ʇṵu}}}}
\dictentry{ɴʇum}{ᵑǀmː}{\dictsense{card.\-num.}{two}{}}
\lettersection{ɴʇʼ}\dictentry{oɴʇʼa}{ɔ̃ᵑǀˀa}{\dictsense{v.}{envelop}{}}
\lettersection{sʇ}\dictentry{sʇau}{s̪ǀau}{\dictsense{adj.}{strong}{}}
\dictentry{sʇui}{s̪ǀui}{\dictsense{n.}{snake}{}}
\lettersection{sʇʼ}\dictentry{sʇʼe}{s̪ǀˀe}{\dictsense{card.\-num.}{four}{}}
\dictentry{sʇʼi}{s̪ǀˀi}{\dictsense{clf.}{slender}{}}
\lettersection{ʘ}\dictentry{ʘui}{ʘui}{\dictsense{v.\-tr.}{want}{}}
\lettersection{ɴʘ}\dictentry{ɴʘaã}{ᵑʘãː}{\dictsense{n.}{air}{}}
\dictentry{ɴʘṵm}{ᵑʘm̰ː}{\dictsense{Ditransitive verb}{(A) suck (substance (T)) from (P), (A) suck on (P)}{}\dictsensesep\dictsense{v.\-tr.}{suckle on, (A) be breastfed by (P)}{\dictexample{ɴʘùm Qʼoaʇùu ǁàa}{Young Qʼoaʇùu is breastfeeding (her child).}}\dictsensesep\dictsense{v.\-tr.}{kiss (on the lips or otherwise)}{}}
\lettersection{cʼ}\dictentry{cʼaã}{cʼãː}{\dictsense{n.}{arm}{\dictclassifiers{\dictref{ʇṵu}}}\dictsensesep\dictsense{n.}{wing (of bird)}{\dictclassifiers{\dictref{ʇṵu}}}}
\dictentry{cʼi}{cʼi}{\dictsense{adj.}{such, similar, of the same kind, of the type being discussed}{}}
\lettersection{j}\dictentry{aje}{aɟe}{\dictsense{clf.}{objects and phenomena in the sky, stars, the sun, the moon, comets, clouds, rainbows, sunrises and sunsets, eclipses, etc.}{}}
\dictentry{ja}{ɟa}{\dictsense{pron.}{1S.INTR}{}}
\dictentry{jaṭa}{ɟaʈa}{\dictsense{pers.\-pr.}{1.PL.EXCL.INTR}{}}
\dictentry{ji}{ɟi}{\dictsense{clf.}{children, young people}{}}
\dictentry{jipa}{ɟipa}{\dictsense{n.}{hare}{}}
\dictentry{jḛeñi}{ɟḛːɲi}{\dictsense{N/A}{this}{}}
\dictentry{jṵm}{ɟm̰ː}{\dictsense{v.\-tr.}{surround, circle, flank, in either threatening or protective manner}{}\dictsensesep\dictsense{n.}{circular formation of people or objects, ring of items, a group encircling a centre}{\dictclassifiers{\dictref{łǁoi}}}\dictsensesep\dictsense{n.}{group of people by a campfire}{}}
\dictentry{uji}{uɟi}{\dictsense{preverb}{ANTIP}{}}
\lettersection{k}\dictentry{ku}{ku}{\dictsense{clf.}{male adults, men}{}}
\dictentry{kuñe}{kuɲe}{\dictsense{adv.}{simply}{}}
\lettersection{l}\dictentry{laã}{lãː}{\dictsense{n.}{tongue}{\dictclassifiers{\dictref{ʇṵu}}}}
\dictentry{la̰a}{la̰ː}{\dictsense{clf.}{foul-smelling objects, rotting matter, feces, waste, corpses, infections, pus, diseases, the severely diseased}{}}
\dictentry{loõṇi}{lɔ̃ːɳi}{\dictsense{adj.}{warm, warming}{}\dictsensesep\dictsense{n.}{warmth}{}\dictsensesep\dictsense{adj.}{sensual, seductive, comforting}{}}
\lettersection{m}\dictentry{mau}{mau}{\dictsense{v.\-intr.}{talk}{}\dictsensesep\dictsense{v.\-intr.}{(O) act like (A), makes decision or behaves according to what is expected of (A)}{}}
\dictentry{maã}{mãː}{\dictsense{clf.}{weather phenomena, winds, rains, sandstorms}{}}
\dictentry{uma}{uma}{\dictsense{pers.\-pr.}{1.PL.INCL.ABS}{}}
\lettersection{p}\dictentry{pau}{pau}{\dictsense{adj.}{abundant}{}}
\dictentry{po}{pɔ}{\dictsense{n.}{lips (of the mouth)}{\dictclassifiers{\dictref{ʇṵu}}}\dictsensesep\dictsense{n.}{mouth, oral cavity}{\dictclassifiers{\dictref{ʇaõ}}}}
\dictentry{upa}{upa}{\dictsense{preverb}{SUBJ}{}}
\lettersection{qʼ}\dictentry{qʼañi}{qʼaɲi}{\dictsense{n.}{vagabond}{}}
\lettersection{r}\dictentry{ra}{ra}{\dictsense{post.}{INSTR}{}}
\lettersection{tł}\dictentry{tła}{t͡ɬa}{\dictsense{clf.}{adult women}{}}
\dictentry{tła̰ʼa}{t͡ɬa̰ʔa}{\dictsense{n.}{fish}{}}
\dictentry{tłḭi}{t͡ɬḭː}{\dictsense{preverb}{COND}{}}
\lettersection{tłʼ}\dictentry{tłʼoi}{t͡ɬʼɔi}{\dictsense{v.\-intr.}{leave, abandon (ABL), exit}{}\dictsensesep\dictsense{v.\-intr.}{(in SVC) stop, interrupt ..., finish, conclude ...}{}}
\dictentry{utłʼe}{ut͡ɬʼe}{\dictsense{n.}{walking path, paved path, dirt road}{}\dictsensesep\dictsense{n.}{groove, incision, indented strip}{}}
\lettersection{tš}\dictentry{tša}{t͡ʃa}{\dictsense{conj.}{while}{}}
\dictentry{tši}{t͡ʃi}{\dictsense{clf.}{aquatic.animals}{}}
\dictentry{tšḛʼe}{t͡ʃḛʔe}{\dictsense{n.}{(of tree) trunk, log, column}{}}
\dictentry{tšḭitši}{t͡ʃḭːt͡ʃi}{\dictsense{v.}{shine}{}}
\dictentry{utši}{ut͡ʃi}{\dictsense{pers.\-pr.}{2.INTR}{}}
\lettersection{tṣ}\dictentry{tṣe}{ʈ͡ʂe}{\dictsense{n.}{peak}{}}
\dictentry{tṣui}{ʈ͡ʂui}{\dictsense{n.}{nose}{}}
\lettersection{ñ}\dictentry{eña}{eɲa}{\dictsense{v.\-tr.}{surrender (smth.), let go of, unwillingly offer}{}}
\dictentry{eña ṣǂqʼo}{eɲa ʂǂ͡qʼɔ}{\dictsense{v.\-tr.}{(A) surrender oneself, give up (lit. offer neck)}{}}
\dictentry{iñi}{iɲi}{\dictsense{post.}{over, on top of, above}{}\dictsensesep\dictsense{Subordinating connective}{provided that, resting on the fact that, the fact that ... guarantees that ...}{}\dictsensesep\dictsense{n.}{head}{}}
\dictentry{ña̰ã}{ɲã̰ː}{\dictsense{N/A}{liver}{}}
\dictentry{ñḛʼe}{ɲḛʔe}{\dictsense{clf.}{elder men}{}}
\lettersection{ŋ}\dictentry{ŋa̰ã}{ŋã̰ː}{\dictsense{n.}{woman}{}}
\dictentry{ŋum}{ŋmː}{\dictsense{card.\-num.}{twelve}{}}
\dictentry{ŋḛe}{ŋḛː}{\dictsense{n.}{evening, time of sunset}{\dictclassifiers{\dictref{ʼurri}}}\dictsensesep\dictsense{n.}{sunset (the process of sun setting)}{}}
\lettersection{ʼ}\dictentry{ʼa}{ʔa}{\dictsense{post.}{ERG}{}\dictsensesep\dictsense{post.}{ABL, coming from, originating from, created by, moving away from}{}}
\dictentry{ʼai}{ʔai}{\dictsense{adv.}{and (for clauses)}{}\dictsensesep\dictsense{adv.}{Back then, in that time, once upon a time}{}}
\dictentry{ʼai ʼai}{ʔai ʔai}{\dictsense{conj.}{and thus, and as a consequence, and immediately after}{}}
\dictentry{ʼa̰o}{ʔa̰ɔ̰}{\dictsense{n.}{water}{\dictclassifiers{\dictref{ǂṵm}}}}
\dictentry{ʼu}{ʔu}{\dictsense{post.}{of}{}}
\dictentry{ʼurri}{ʔurri}{\dictsense{clf.}{timespans, events in time, occurrences, dates, appointments}{}}
\dictentry{ʼutła}{ʔut͡ɬa}{\dictsense{n.}{playing ball}{\dictclassifiers{\dictref{ṉoõ}}}}
\dictentry{ʼḛe}{ʔḛː}{\dictsense{post.}{PTV}{}}
\dictentry{ʼḭi}{ʔḭː}{\dictsense{clf.}{slithering.animals}{}}
\dictentry{ʼṵa}{ʔṵa̰}{\dictsense{v.\-tr.}{(someone) meet, make acquaintance of, get to know, greet, receive}{}}
\lettersection{ṇ}\dictentry{ṇam}{ɳãm}{\dictsense{conj.}{but}{}}
\dictentry{ṇa̰ã}{ɳã̰ː}{\dictsense{n.}{eland}{\dictclassifiers{\dictref{ǂʼa}}}}
\dictentry{ṇṵĩ}{ɳm̰ḭ̃}{\dictsense{clf.}{spoken word, utterances, phrases, languages, words, voices, thoughts, reasonings, stories}{}}
\lettersection{ṉ}\dictentry{ṉa}{ṉa}{\dictsense{n.}{grass}{}\dictsensesep\dictsense{n.}{hair, head hair, facial hair, body hair}{}\dictsensesep\dictsense{n.}{(of animal) fur}{}}
\dictentry{ṉa̰ã}{ṉã̰ː}{\dictsense{v.}{laugh}{}}
\dictentry{ṉoõ}{ṉɔ̃ː}{\dictsense{clf.}{round tools, round instruments, artificial balls, spheres, globes, round toys}{}}
\dictentry{ṉoṯi}{ṉɔt͡s̪i}{\dictsense{card.\-num.}{eight}{}}
\dictentry{ṉui}{ṉui}{\dictsense{clf.}{persons, people, humans, personified entities, individuals, animate}{}}
\lettersection{ṭ}\dictentry{uṭu}{uʈu}{\dictsense{clf.}{bird}{}}
\dictentry{ṭa̰a}{ʈa̰ː}{\dictsense{clf.}{predatory animals, carnivores}{}}
\dictentry{ṭuma}{ʈuma}{\dictsense{adj.}{great, awesome}{}}
\dictentry{ṭurra}{ʈurra}{\dictsense{adv.}{suddenly}{}}
\dictentry{ṭḛe}{ʈḛː}{\dictsense{clf.}{articles of clothing, cloth, shoes}{}}
\lettersection{ṭʼ}\dictentry{ṭʼaṇi}{ʈʼaɳi}{\dictsense{v.\-tr.}{remind}{}}
\dictentry{ṭʼoa}{ʈʼɔa}{\dictsense{n.}{corpse, cadaver, carcass (human or animal)}{\dictclassifiers{\dictref{la̰a}}}}
\lettersection{ṯ}\dictentry{eṯe}{et͡s̪e}{\dictsense{N/A}{when}{}}
\dictentry{ṯui}{t͡s̪ui}{\dictsense{v.\-intr.}{cry}{}}
\dictentry{ṯṵu}{t͡s̪ṵː}{\dictsense{v.\-tr.}{throw, launch}{}\dictsensesep\dictsense{v.\-tr.}{produce, spit out, blow, excrete}{}}

\end{singlespace}
\end{multicols}


\end{document}